\documentclass[a4paper,11pt]{article}
\usepackage[utf8]{inputenc}
% ------
% LAYOUT
% ------
\textwidth 165mm %
\textheight 230mm %
\oddsidemargin 0mm %
\evensidemargin 0mm %
\topmargin -15mm %
\parindent= 10mm
\setlength{\parskip}{0.5em}

\usepackage[dvips]{graphics}
\usepackage[table]{xcolor}
\usepackage{listings}
\usepackage{color}
\usepackage{graphicx}
% \usepackage{subfigure}
\usepackage{amssymb}
\usepackage{amsfonts}
\usepackage{amsthm}
\usepackage{amsmath}
\usepackage{mathtools}
\usepackage{pifont}
\usepackage[outdir=./]{epstopdf}
\usepackage{caption}
\usepackage{subcaption}
% 
% \usepackage[framed,numbered,autolinebreaks,useliterate]{mcode}

% My short functions
\newcommand{\V}[1]{\boldsymbol{#1}}
\newcommand{\floor}[1]{\lfloor #1 \rfloor}
\newcommand{\mat}[1]{\begin{bmatrix}#1\end{bmatrix}}
\newcommand{\refEq}[1]{Eq. \ref{#1}}
\newcommand{\reflst}[1]{Listing \ref{#1} at page \pageref{#1}}
\newcommand{\reftbl}[1]{Table \ref{#1}}
\newcommand{\reffig}[1]{Figure \ref{#1}}
\newcommand{\dx}[2]{\frac{\partial #1}{\partial #2}}
\newcommand{\ddx}[2]{\frac{\partial^2 #1}{\partial #2^2}}
\newcommand{\cmark}{{\color{blue}\text{\ding{51}}}}%
\newcommand{\xmark}{{\color{red}\text{\ding{55}}}}%
\newcommand{\TODO}{{\color{red}TODO}}

%listing styles
\definecolor{mGreen}{rgb}{0,0.6,0}
\definecolor{mGray}{rgb}{0.5,0.5,0.5}
\definecolor{mPurple}{rgb}{0.58,0,0.82}

\lstdefinestyle{CStyle}{
    backgroundcolor=\color{white},   
    commentstyle=\color{mGreen},
    keywordstyle=\color{magenta},
    stringstyle=\color{mPurple},
    basicstyle=\footnotesize,
    breakatwhitespace=false,         
    breaklines=true,                 
    captionpos=t,                    
    keepspaces=true,
    frame=trlb,
    numbers=left,                    
    numbersep=5pt,                  
    showspaces=false,                
    showstringspaces=false,
    showtabs=false,                  
    tabsize=2,
    language=C
}

\definecolor{mygreen}{rgb}{0,0.6,0}
\definecolor{mygray}{rgb}{0.5,0.5,0.5}
\definecolor{mymauve}{rgb}{0.58,0,0.82}
\lstdefinestyle{PythonStyle}{ %
  language = python,
  backgroundcolor=\color{white},   % choose the background color
  basicstyle=\footnotesize,        % size of fonts used for the code
  frame = trlb,
  numbers=left,
  stepnumber=1,
  showstringspaces=false,
  tabsize=1,
  breaklines=true,                 % automatic line breaking only at whitespace
  captionpos=t,                    % sets the caption-position to bottom
  commentstyle=\color{mygreen},    % comment style
  %escapeinside={\%*}{*)},          % if you want to add LaTeX within your code
  keywordstyle=\color{blue},       % keyword style
  stringstyle=\color{mymauve},     % string literal style
}
%%###########################

\begin{document}
\begin{center}
\Large{\textbf{ECE 637: Lab 2}}

Rahul Deshmukh

\today
\end{center}

%#########################################
\subsection*{Section 1 Report} 

\begin{enumerate}
\item Gray scale image img04g.tif
\begin{figure}[!hp]
 \centering
 \includegraphics[width=\textwidth]{../pix/img04g}
 \caption{Input gray scale image}
\end{figure}

\newpage
\item Power Spectral density plots:
\begin{figure}[!hp]
 \centering
 \begin{subfigure}[b]{0.9\textwidth}
  \centering
  \includegraphics[width=0.6\textwidth]{../pix/SpecAnal_64}
  \caption{PSD for 64x64}
 \end{subfigure}
\begin{subfigure}[b]{0.9\textwidth}
  \centering
  \includegraphics[width=0.6\textwidth]{../pix/SpecAnal_128}
  \caption{PSD for 128x128}
 \end{subfigure}
 \begin{subfigure}[b]{0.9\textwidth}
   \centering
  \includegraphics[width=0.6\textwidth]{../pix/SpecAnal_256}
  \caption{PSD for 256x256}
 \end{subfigure}
\end{figure}


\newpage
\item Improved PSD using \textit{BetterSpecAnal(x)} function:
\begin{figure}[!hp]
 \centering
 \includegraphics[width=\textwidth]{../pix/BetterSpecAnal_img04g}
 \caption{improved PSD for img04g.tif}
\end{figure}

\item For python code of \textit{BetterSpecAnal(x)} function refer to \reflst{lst:better_spec_anal}.
\end{enumerate}

%
\clearpage
\vspace{2ex}
%
\subsection*{Section 2 Report} 
\begin{enumerate}
\item The image $255*(x+0.5)$:
\begin{figure}[!hp]
 \centering
 \includegraphics[width=\textwidth]{../pix/Random_image}
 \caption{random gray scale image generated}
\end{figure}

We need to find the difference equation for the given transfer function:
\begin{align*}
 H(z_1,z_2)= \frac{Y(z_1,z_2)}{X(z_1,z_2)}&= \frac{3}{1-0.99z_1^{-1} -0.99z_2^{-1}+0.981z_1^{-1}z_2^{-1}}\\
 \Rightarrow Y(z_1,z_2)&= 3X(z_1,z_2) + 0.99(z_1^{-1} + z_2^{-1}) Y(z_1,z_2)  -0.981 z_1^{-1}z_2^{-1} Y(z_1,z_2)\\
 &\text{Therefore the difference equation is given by:}\\
 y(m,n) &= 3x(m,n) + 0.99(y(m-1,n) + y(m,n-1)) -0.981y(m-1,n-1)
\end{align*}

\newpage
\item The image $y+127$:

\begin{figure}[!hp]
 \centering
 \includegraphics[width=\textwidth]{../pix/filtered_image}
 \caption{IIR Filtered image}
\end{figure}


\newpage
\item  Mesh plot of log $S_y(e^{j\mu},e^{j\nu})$:
\begin{figure}[!hp]
 \centering
 \includegraphics[width=\textwidth]{../pix/SpecAnal_filtered_image_64}
 \caption{Mesh plot of PSD of y}
\end{figure}

\newpage
\item Mesh plot of log of estimated PSD of y using \textit{BetterSpecAnal(y)}:
\begin{figure}[!hp]
 \centering
 \includegraphics[width=\textwidth]{../pix/BetterSpecAnal_filtered_image}
 \caption{Mesh plot of PSD of y using \textit{BetterSpecAnal(x)}}
\end{figure}

\end{enumerate}


\newpage
\subsection*{Source Code}
%
\begin{lstlisting}[style={PythonStyle},caption={Python function for BetterSpecAnal},label={lst:better_spec_anal}] 
def BetterSpecAnal(x,base_name):
    block_size=64
    h,w = x.shape
    cx=w//2; cy=h//2
    or_x = cx-5*(block_size//2)
    or_y = cy-5*(block_size//2)
    windows = []
    for i in range(5):
        for j in range(5):
            ul_x = or_x +i*block_size
            ul_y = or_y +j*block_size
            windows.append(x[ul_y:ul_y+block_size,ul_x:ul_x+block_size])
    W = np.hamming(block_size)
    W = np.outer(W,W)
    #multiply 2D hamming window
    windows = [w*W for w in windows]
    #compute squred DFT magnitude
    Z = [ (1/(block_size**2)*np.abs(np.fft.fft2(w))**2) for w in windows]
    Z = [np.fft.fftshift(z) for z in Z]
    Zabs = [np.log(z) for z in Z]
    #compute average
    av = np.zeros((block_size,block_size))
    for z in Zabs: av+=z
    av/=25
    
    # Plot the result using a 3-D mesh plot and label the x and y axises properly. 
    fig = plt.figure()
    ax = fig.add_subplot(111, projection='3d')
    a = b = np.linspace(-np.pi, np.pi, num = block_size)
    X, Y = np.meshgrid(a, b)
    
    surf = ax.plot_surface(X, Y, av, cmap=plt.cm.coolwarm)
    ax.set_xlim(-1*np.pi,1*np.pi)
    ax.set_ylim(-1*np.pi,1*np.pi)
    ax.autoscale(enable=True,axis='z',tight=True)
    ax.set_xlabel('$\mu$ axis')
    ax.set_ylabel('$\\nu$ axis')
    ax.set_zlabel('Z Label')
    
    fig.colorbar(surf, shrink=0.5, aspect=5)
    plt.savefig('BetterSpecAnal_'+base_name+'.eps',format='eps')
\end{lstlisting}

\subsection*{Appendix}
\lstinputlisting[caption={Complete python code for computing PSD},style={PythonStyle}]{../code/MySpecAnal.py}
\lstinputlisting[caption={Python code for generating filtered image for section 2},style={PythonStyle}]{../code/generate_filtered_image.py}
% 
\end{document}

\documentclass[a4paper,11pt]{article}
\usepackage[utf8]{inputenc}
% ------
% LAYOUT
% ------
\textwidth 165mm %
\textheight 230mm %
\oddsidemargin 0mm %
\evensidemargin 0mm %
\topmargin -15mm %
\parindent= 10mm
\setlength{\parskip}{0.5em}

\usepackage[dvips]{graphics}
\usepackage[table]{xcolor}
\usepackage{listings}
\usepackage{color}
\usepackage{graphicx}
% \usepackage{subfigure}
\usepackage{amssymb}
\usepackage{amsfonts}
\usepackage{amsthm}
\usepackage{amsmath}
\usepackage{mathtools}
\usepackage{pifont}
\usepackage[outdir=./]{epstopdf}
\usepackage{caption}
\usepackage{subcaption}
% 
% \usepackage[framed,numbered,autolinebreaks,useliterate]{mcode}

% My short functions
\newcommand{\V}[1]{\boldsymbol{#1}}
\newcommand{\floor}[1]{\lfloor #1 \rfloor}
\newcommand{\mat}[1]{\begin{bmatrix}#1\end{bmatrix}}
\newcommand{\refEq}[1]{Eq. \ref{#1}}
\newcommand{\reflst}[1]{Listing \ref{#1} at page \pageref{#1}}
\newcommand{\reftbl}[1]{Table \ref{#1}}
\newcommand{\reffig}[1]{Figure \ref{#1}}
\newcommand{\dx}[2]{\frac{\partial #1}{\partial #2}}
\newcommand{\ddx}[2]{\frac{\partial^2 #1}{\partial #2^2}}
\newcommand{\cmark}{{\color{blue}\text{\ding{51}}}}%
\newcommand{\xmark}{{\color{red}\text{\ding{55}}}}%
\newcommand{\TODO}{{\color{red}TODO}}

%listing styles
\definecolor{mGreen}{rgb}{0,0.6,0}
\definecolor{mGray}{rgb}{0.5,0.5,0.5}
\definecolor{mPurple}{rgb}{0.58,0,0.82}

\lstdefinestyle{CStyle}{
    backgroundcolor=\color{white},   
    commentstyle=\color{mGreen},
    keywordstyle=\color{magenta},
    stringstyle=\color{mPurple},
    basicstyle=\footnotesize,
    breakatwhitespace=false,         
    breaklines=true,                 
    captionpos=t,                    
    keepspaces=true,
    frame=trlb,
    numbers=left,                    
    numbersep=5pt,                  
    showspaces=false,                
    showstringspaces=false,
    showtabs=false,                  
    tabsize=2,
    language=C
}

\definecolor{mygreen}{rgb}{0,0.6,0}
\definecolor{mygray}{rgb}{0.5,0.5,0.5}
\definecolor{mymauve}{rgb}{0.58,0,0.82}
\lstdefinestyle{PythonStyle}{ %
  language = python,
  backgroundcolor=\color{white},   % choose the background color
  basicstyle=\footnotesize,        % size of fonts used for the code
  frame = trlb,
  numbers=left,
  stepnumber=1,
  showstringspaces=false,
  tabsize=1,
  breaklines=true,                 % automatic line breaking only at whitespace
  captionpos=t,                    % sets the caption-position to bottom
  commentstyle=\color{mygreen},    % comment style
  %escapeinside={\%*}{*)},          % if you want to add LaTeX within your code
  keywordstyle=\color{blue},       % keyword style
  stringstyle=\color{mymauve},     % string literal style
}
%%###########################

\begin{document}
\begin{center}
\Large{\textbf{ECE 637: Lab 1}}

Rahul Deshmukh

\today
\end{center}

%#########################################
\subsection*{Section 3 Report} 

\begin{enumerate}
\item Derivation of $H(e^{j\mu},e^{j\nu})$:
\begin{align*}
 h(m,n) &= 
 \begin{cases}
 \frac{1}{81} & |m|\leq 4, |n|\leq 4\\
 0 & \text{elsewhere}
 \end{cases}\\
 \text{Let}\quad h_1(m) &= \frac{1}{9}\sum_{i=-4}^4 \delta(m-i)\\
 \text{Then}\quad h(m,n)&=h_1(m)h_1(n)\\
 \Rightarrow H_1(e^{ju}) &= \frac{1}{9}\sum_{i=-4}^4 e^{-jiu}\\
 &= \frac{1}{9} (1+ e^{-ju} + e^{-2ju}+ e^{-3ju} + e^{-4ju} +
 e^{ju} + e^{2ju}+ e^{3ju} + e^{4ju})\\
 &= \frac{1}{9}(1+ 2(\sum_{i=1}^{4}cos(iu)) )\\
 \Rightarrow H(e^{ju},e^{jv}) &= H_1(e^{ju}) H_1(e^{jv})\\
 &= \frac{1}{81}(1+ 2(\sum_{i=1}^{4}cos(iu)))(1+ 2(\sum_{k=1}^{4}cos(kv)) )
\end{align*}



\newpage
\item Plot of $|H(e^{j\mu},e^{j\nu})|$
\begin{figure}[!hp]
 \centering
 \includegraphics[width=\textwidth]{../code/python/fig1}
 \caption{Surface plot}
\end{figure}

\begin{figure}[!hp]
 \centering
 \includegraphics[width=0.5\textwidth]{../code/python/fig1_contour}
 \caption{Contour plot}
\end{figure}

\newpage
\item Color image of img03.tif
\begin{figure}[!hp]
 \centering
 \includegraphics[width=0.5\textwidth]{../pix/img03}
 \caption{color image img03.tif}
\end{figure}

\newpage
\item FIR Filtered image of img03.tif
\begin{figure}[!hp]
 \centering
 \includegraphics[width=0.5\textwidth]{../code/C-Code/run/output/img03_FIR}
 \caption{FIR filtered color image}
 \end{figure}

 \item For C-code for FIR filter refer to \reflst{lst:fir}.
\end{enumerate}

%
\clearpage
\vspace{2ex}
%
\subsection*{Section 4 Report} 
\begin{enumerate}
\item Derivation of $H(e^{j\mu},e^{j\nu})$:
\begin{align*}
 h(m,n) &= 
 \begin{cases}
 \frac{1}{25} & |m|\leq 2, |n|\leq 2\\
 0 & \text{elsewhere}
 \end{cases}\\
 \text{Let}\quad h_1(m) &= \frac{1}{5}\sum_{i=-2}^2 \delta(m-i)\\
 \text{Then}\quad h(m,n)&=h_1(m)h_1(n)\\
 \Rightarrow H_1(e^{ju}) &= \frac{1}{5}\sum_{i=-2}^2 e^{-jiu}\\
 &= \frac{1}{5} (1+ e^{-ju} + e^{-2ju}+  e^{ju} + e^{2ju})\\
 &= \frac{1}{5}(1+ 2(\sum_{i=1}^{2}cos(iu)) )\\
 \Rightarrow H(e^{ju},e^{jv}) &= H_1(e^{ju}) H_1(e^{jv})\\
 &= \frac{1}{25}(1+ 2(\sum_{i=1}^{2}cos(iu)))(1+ 2(\sum_{k=1}^{2}cos(kv)) )
\end{align*}

\item Derivation of $G(e^{j\mu},e^{j\nu})$:
\begin{align*}
 g(m,n)&=\delta(m,n) +\lambda(\delta(m,n)-h(m,n))\\
 G(e^{ju},e^{jv}) &= 1 + \lambda(1-H(e^{ju},e^{jv}))\\
 &= 1+ \lambda(1- \frac{1}{25}(1+ 2(\sum_{i=1}^{2}cos(iu)))(1+ 2(\sum_{k=1}^{2}cos(kv)) ) )
\end{align*}



\newpage
\item Plot of $|H(e^{j\mu},e^{j\nu})|$
\begin{figure}[!hp]
 \centering
 \includegraphics[width=\textwidth]{../code/python/fig2}
 \caption{Surface plot}
\end{figure}

\begin{figure}[!hp]
 \centering
 \includegraphics[width=0.5\textwidth]{../code/python/fig2_contour}
 \caption{Contour plot}
\end{figure}

\newpage
\item Plot of $|G(e^{j\mu},e^{j\nu})|$ for $\lambda=1.5$
\begin{figure}[!hp]
 \centering
 \includegraphics[width=\textwidth]{../code/python/fig3}
 \caption{Surface plot}
\end{figure}

\begin{figure}[!hp]
 \centering
 \includegraphics[width=0.5\textwidth]{../code/python/fig3_contour}
 \caption{Contour plot}
\end{figure}

\newpage
\item Input color image imgblur.tif
\begin{figure}[!hp]
 \centering
 \includegraphics[width=0.5\textwidth]{../pix/imgblur}
 \caption{color image imgblur.tif}
\end{figure}

\newpage
\item Output sharpened image for $\lambda=1.5$
\begin{figure}[!hp]
\centering
\includegraphics[width=0.5\textwidth]{../code/C-Code/run/output/imgblur_FIR_sharp}
\caption{output sharpened image}
\end{figure}

\item For C-code for FIR sharpening filter refer to \reflst{lst:fir_sharpen}.
\end{enumerate}

%
\clearpage
\vspace{2ex}
%
\subsection*{Section 5 Report}
\begin{enumerate}
\item Derivation of $H(e^{j\mu},e^{j\nu})$:
\begin{align*}
 y(m,n) &= 0.01x(m,n) + 0.9(y(m-1,n)+y(m,n-1)) -0.81y(m-1,n-1)\\
 &\text{Taking Z transform we get}\\
 Y(z_1,z_2)&= 0.01X(z_1,z_2) + 0.9(z_1^{-1}Y(z_1,z_2)+z_2^{-1}Y(z_1,z_2)) -0.81z_1^{-1}z_2^{-1}Y(z_1,z_2)\\
 \Rightarrow H(z_1,z_2) &= \frac{Y(z_1,z_2)}{X(z_1,z_2)}\\
 &= \frac{0.01}{0.9(z_1^{-1} + z_2^{-1}) -0.81z_1^{-1}z_2^{-1}}\Big|_{z_1=e^{ju},z_2=e^{jv}}\\
 \Rightarrow H(e^{ju},e^{jv}) &= \frac{0.01}{0.9(e^{-ju} + e^{-jv}) -0.81e^{-ju}e^{-jv}}
\end{align*}

\newpage
\item Plot of $|H(e^{j\mu},e^{j\nu})|$
\begin{figure}[!hp]
 \centering
 \includegraphics[width=\textwidth]{../code/python/fig4}
 \caption{Surface plot}
\end{figure}

\begin{figure}[!hp]
 \centering
 \includegraphics[width=0.5\textwidth]{../code/python/fig4_contour}
 \caption{Contour plot}
\end{figure}

\item Image of the point spread functions
% \begin{figure}[!hp]
%  \centering
%  \includegraphics[width=0.5\textwidth]{../code/python/lab1_52}
%  \caption{Contour plot}
% \end{figure}

\begin{figure}[!hp]
 \centering
 \begin{subfigure}[b]{0.45\textwidth}
  \centering
  \includegraphics[width=\textwidth]{../code/python/lab1_52.jpg}
  \caption{image for $x(m,n)=\delta(m-127,n-127)$}
 \end{subfigure}
\hfill
\begin{subfigure}[b]{0.45\textwidth}
  \centering
  \includegraphics[width=\textwidth]{../code/python/lab1_52_IIR}
  \caption{IIR filtered image $y(m,n)$}
 \end{subfigure}
\end{figure}


\newpage
\item IIR Filtered output color image
\begin{figure}[!hp]
\centering
\includegraphics[width=0.5\textwidth]{../code/C-Code/run/output/img03_IIR}
\caption{IIR filtered image}
\end{figure}


\item For C-code for IIR filter refer to \reflst{lst:iir}.
\end{enumerate}



\newpage
\subsection*{Source Code}
%
\lstinputlisting[caption={FIR},label={lst:fir}, style=CStyle]{../code/C-Code/src/ImageReadWrite_FIR.c}
\lstinputlisting[caption={FIR sharpening},label={lst:fir_sharpen}, style=CStyle]{../code/C-Code/src/ImageReadWrite_FIR_sharpen.c}
\lstinputlisting[caption={IIR},label={lst:iir}, style=CStyle]{../code/C-Code/src/ImageReadWrite_IIR.c}

\subsection*{Appendix}
\lstinputlisting[caption={Python code for frequency plots},label={lst:py_fft}, style=PythonStyle]{../code/python/lab1.py}

\lstinputlisting[caption={Python code for plotting psf for IIR filter},label={lst:py_psf_iir}, style=PythonStyle]{../code/python/ex5.2.py}

\begin{lstlisting} 
\end{lstlisting}
% 
\end{document}

\documentclass[a4paper,11pt]{article}

% ------
% LAYOUT
% ------
\textwidth 165mm %
\textheight 230mm %
\oddsidemargin 0mm %
\evensidemargin 0mm %
\topmargin -15mm %
\parindent= 10mm
\setlength{\parskip}{0.5em}

\usepackage[utf8]{inputenc}
\usepackage[T1]{fontenc}
\usepackage[dvips]{graphics}
\usepackage[table]{xcolor}
\usepackage{listings}
\usepackage{color}
\usepackage{graphicx}
% \usepackage{subfigure}
\usepackage{amssymb}
\usepackage{amsfonts}
\usepackage{amsthm}
\usepackage{amsmath}
\usepackage{mathtools}
\usepackage{pifont}
\usepackage[outdir=./]{epstopdf}
\usepackage{caption}
\usepackage{subcaption}
\usepackage{hyperref}
\usepackage{textcomp}
% 
% \usepackage[framed,numbered,autolinebreaks,useliterate]{mcode}
\hypersetup{
    colorlinks=true,
    linkcolor=blue,
    filecolor=magenta,      
    urlcolor=cyan,
}

\urlstyle{same}

% My short functions
\newcommand{\V}[1]{\boldsymbol{#1}}
\newcommand{\floor}[1]{\lfloor #1 \rfloor}
\newcommand{\mat}[1]{\begin{bmatrix}#1\end{bmatrix}}
\newcommand{\refEq}[1]{Eq. \ref{#1}}
\newcommand{\reflst}[1]{Listing \ref{#1} at page \pageref{#1}}
\newcommand{\reftbl}[1]{Table \ref{#1}}
\newcommand{\reffig}[1]{Figure \ref{#1}}
\newcommand{\dx}[2]{\frac{\partial #1}{\partial #2}}
\newcommand{\ddx}[2]{\frac{\partial^2 #1}{\partial #2^2}}
\newcommand{\cmark}{{\color{blue}\text{\ding{51}}}}%
\newcommand{\xmark}{{\color{red}\text{\ding{55}}}}%
\newcommand{\TODO}{{\color{red}TODO}}

%listing styles
\definecolor{mGreen}{rgb}{0,0.6,0}
\definecolor{mGray}{rgb}{0.5,0.5,0.5}
\definecolor{mPurple}{rgb}{0.58,0,0.82}
\definecolor{mygreen}{rgb}{0,0.6,0}
\definecolor{mygray}{rgb}{0.5,0.5,0.5}
\definecolor{mymauve}{rgb}{0.58,0,0.82}

\lstset{upquote=true}

\lstdefinestyle{CStyle}{
    backgroundcolor=\color{white},   
    commentstyle=\color{mGreen},
    keywordstyle=\color{magenta},
    stringstyle=\color{mPurple},
    basicstyle=\footnotesize,
    breakatwhitespace=false,         
    breaklines=true,                 
    captionpos=t,                    
    keepspaces=true,
    frame=trlb,
    numbers=left,                    
    numbersep=5pt,                  
    showspaces=false,                
    showstringspaces=false,
    showtabs=false,                  
    tabsize=2,
    language=C
}

\lstdefinestyle{TextStyle}{
    backgroundcolor=\color{white},   
%     stringstyle=\color{mPurple},
    basicstyle=\footnotesize,
    breakatwhitespace=false,         
    breaklines=true,                 
    captionpos=t,                    
    keepspaces=true,
    frame=trlb,
    numbers=left,                    
    numbersep=5pt,                  
    showspaces=false,                
    showstringspaces=false,
    showtabs=false,                  
    tabsize=2,
}

\lstdefinestyle{PythonStyle}{ %
  language = python,
  backgroundcolor=\color{white},   % choose the background color
  basicstyle=\footnotesize,        % size of fonts used for the code
  frame = trlb,
  numbers=left,
  stepnumber=1,
  showstringspaces=false,
  tabsize=1,
  breaklines=true,                 % automatic line breaking only at whitespace
  captionpos=t,                    % sets the caption-position to bottom
  commentstyle=\color{mygreen},    % comment style
  %escapeinside={\%*}{*)},          % if you want to add LaTeX within your code
  keywordstyle=\color{blue},       % keyword style
  stringstyle=\color{mymauve},     % string literal style
}

\lstdefinestyle{BashStyle}{ %
  language = bash,
  backgroundcolor=\color{white},   % choose the background color
  basicstyle=\footnotesize,        % size of fonts used for the code
  frame = trlb,
  numbers=left,
  stepnumber=1,
  showstringspaces=false,
  tabsize=1,
  breaklines=true,                 % automatic line breaking only at whitespace
  captionpos=t,                    % sets the caption-position to bottom
  commentstyle=\color{blue},    % comment style
  %escapeinside={\%*}{*)},          % if you want to add LaTeX within your code
  keywordstyle=\color{orange},       % keyword style
  stringstyle=\color{mymauve},     % string literal style
}
%%###########################

\begin{document}
\begin{center}
\Large{\textbf{ECE 637: Lab 8}}

Rahul Deshmukh\\\href{mailto:deshmuk5@purdue.edu}{{\color{blue}deshmuk5@purdue.edu}}

\today
\end{center}

%#########################################
\subsection*{Section 3.1 Report} 
\begin{enumerate}
\item Original and thresholded image:
\begin{figure}[!hp]
  \centering
   \begin{subfigure}{0.45\textwidth}
  \includegraphics[width=\textwidth]{../pix/house}
 \caption{Original image}
 \end{subfigure}
  \begin{subfigure}{0.45\textwidth}
  \includegraphics[width=\textwidth]{../code/python/output/sec3/house_threshold}
 \caption{Thresholded image}
 \end{subfigure}
  
 \end{figure}
\item Computed RMSE and fidelity values:
    \lstinputlisting[caption={RMSE and fidelity for section 3.1},style={TextStyle},label={lst:dither_mats}]{../code/python/output/sec3/sec3.log}

\item For python code for fidelity function refer to \reflst{lst:utils}  
\end{enumerate}
 
For python code refer to \reflst{lst:sec3} and \reflst{lst:bash}.

\clearpage
\subsection*{Section 4.2 Report} 
\begin{itemize}
\item The three Bayer index matrices of sizes 2x2, 4x4, and 8x8:
    \lstinputlisting[caption={Bayer index matrices},style={TextStyle},label={lst:dither_mats}]{../code/python/output/sec4/dither_mats.log}
 
\item The three halftoned images:
\begin{figure}[!hp]
  \centering
 \begin{subfigure}{0.3\textwidth}
  \includegraphics[width=\textwidth]{../code/python/output/sec4/house_dither2}
 \caption{Halftoned image for 2x2}
 \end{subfigure}
 \begin{subfigure}{0.3\textwidth}
  \includegraphics[width=\textwidth]{../code/python/output/sec4/house_dither4}
 \caption{Halftoned image for 4x4}
 \end{subfigure}
 \begin{subfigure}{0.3\textwidth}
  \includegraphics[width=\textwidth]{../code/python/output/sec4/house_dither8}
 \caption{Halftoned image for 8x8}
 \end{subfigure}
 \end{figure}
 
\item RMSE and Fidelity for each of the three halftoned images:
 \lstinputlisting[caption={RMSE and fidelity for 2x2},style={TextStyle},label={lst:rmse2}]{../code/python/output/sec4/dither_2.log}
 \lstinputlisting[caption={RMSE and fidelity for 4x4},style={TextStyle},label={lst:rmse4}]{../code/python/output/sec4/dither_4.log}
 \lstinputlisting[caption={RMSE and fidelity for 8x8},style={TextStyle},label={lst:rmse8}]{../code/python/output/sec4/dither_8.log}
\end{itemize} 
For python refer to \reflst{lst:sec4}, \reflst{lst:sec4print} and \reflst{lst:bash}.

\clearpage
\subsection*{Section 5 Report} 
\begin{itemize}
 \item For python code refer to \reflst{lst:sec5py}
 \item The error diffusion result:
 \begin{figure}[!hp]
  \centering
  \includegraphics[width=0.5\textwidth]{../code/python/output/sec5/house_error_diff}
  \caption{Error diffusion result}
 \end{figure}
 \item RMSE and fidelity results:
  \lstinputlisting[caption={RMSE and fidelity},style={TextStyle},label={lst:sec5log}]{../code/python/output/sec5/sec5.log}
\item Table of RMSE and fidelity
\begin{table}[!h]
    \centering
    \begin{tabular}{|c|c|c|}
    \hline
    Method& RMSE& Fidelity\\
    \hline
    Simple thresholding& 87.393& 75.420\\
    \hline
    Order dithering 2x2& 97.669& 48.217\\
    \hline
    Order dithering 4x4& 101.007& 15.524\\
    \hline
    Order dithering 8x8& 100.915& 15.087\\
    \hline
    Error diffusion& 98.847& 13.640\\
    \hline
    \end{tabular}
\end{table}

As we can observe visually from the halftoned images produced using different methods, the image contours improved when we used order dithering. However, the RMSE metric increases in this case which indicates that RMSE is not an appropriate metric. For the case of Fidelity the metric score decreases as we increase the size of dithering and visually the image contours improve and we can see smoother grayscale colors. Therefore, fidelity is a good metric in this case. Fidelity score is the lowest for Error diffusion method and visually the image also looks the closest to the input image out of all the halftoned images.

\end{itemize}


\clearpage
\subsection*{Appendix}
Got to \href{https://github.com/rahuldeshmukh43/Courses/tree/master/ECE637-DigitalImageProcessing-I/labs/lab8/code/}{git repo} for complete code.
%python code
\lstinputlisting[caption={Python code for section 3},style={PythonStyle},label={lst:sec3}]{../code/python/simple_thresholding.py}
\lstinputlisting[caption={Python code for RMSE and Fidelity},style={PythonStyle},label={lst:utils}]{../code/python/utils.py}
\lstinputlisting[caption={Python code for section 4},style={PythonStyle},label={lst:sec4}]{../code/python/order_dithering.py}
\lstinputlisting[caption={Python code for printing bayer matrices},style={PythonStyle},label={lst:sec4print}]{../code/python/print_dither.py}
\lstinputlisting[caption={Python code for section 5},style={PythonStyle},label={lst:sec5py}]{../code/python/error_diffusion.py}
%bash code
\lstinputlisting[caption={Bash code for running python code},style={BashStyle},label={lst:bash}]{../code/python/run.sh}

\end{document}

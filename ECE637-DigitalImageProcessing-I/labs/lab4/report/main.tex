\documentclass[a4paper,11pt]{article}

% ------
% LAYOUT
% ------
\textwidth 165mm %
\textheight 230mm %
\oddsidemargin 0mm %
\evensidemargin 0mm %
\topmargin -15mm %
\parindent= 10mm
\setlength{\parskip}{0.5em}

\usepackage[utf8]{inputenc}
\usepackage[T1]{fontenc}
\usepackage[dvips]{graphics}
\usepackage[table]{xcolor}
\usepackage{listings}
\usepackage{color}
\usepackage{graphicx}
% \usepackage{subfigure}
\usepackage{amssymb}
\usepackage{amsfonts}
\usepackage{amsthm}
\usepackage{amsmath}
\usepackage{mathtools}
\usepackage{pifont}
\usepackage[outdir=./]{epstopdf}
\usepackage{caption}
\usepackage{subcaption}
\usepackage{hyperref}
\usepackage{textcomp}
% 
% \usepackage[framed,numbered,autolinebreaks,useliterate]{mcode}
\hypersetup{
    colorlinks=true,
    linkcolor=blue,
    filecolor=magenta,      
    urlcolor=cyan,
}

\urlstyle{same}

% My short functions
\newcommand{\V}[1]{\boldsymbol{#1}}
\newcommand{\floor}[1]{\lfloor #1 \rfloor}
\newcommand{\mat}[1]{\begin{bmatrix}#1\end{bmatrix}}
\newcommand{\refEq}[1]{Eq. \ref{#1}}
\newcommand{\reflst}[1]{Listing \ref{#1} at page \pageref{#1}}
\newcommand{\reftbl}[1]{Table \ref{#1}}
\newcommand{\reffig}[1]{Figure \ref{#1}}
\newcommand{\dx}[2]{\frac{\partial #1}{\partial #2}}
\newcommand{\ddx}[2]{\frac{\partial^2 #1}{\partial #2^2}}
\newcommand{\cmark}{{\color{blue}\text{\ding{51}}}}%
\newcommand{\xmark}{{\color{red}\text{\ding{55}}}}%
\newcommand{\TODO}{{\color{red}TODO}}

%listing styles
\definecolor{mGreen}{rgb}{0,0.6,0}
\definecolor{mGray}{rgb}{0.5,0.5,0.5}
\definecolor{mPurple}{rgb}{0.58,0,0.82}
\definecolor{mygreen}{rgb}{0,0.6,0}
\definecolor{mygray}{rgb}{0.5,0.5,0.5}
\definecolor{mymauve}{rgb}{0.58,0,0.82}

\lstset{upquote=true}

\lstdefinestyle{CStyle}{
    backgroundcolor=\color{white},   
    commentstyle=\color{mGreen},
    keywordstyle=\color{magenta},
    stringstyle=\color{mPurple},
    basicstyle=\footnotesize,
    breakatwhitespace=false,         
    breaklines=true,                 
    captionpos=t,                    
    keepspaces=true,
    frame=trlb,
    numbers=left,                    
    numbersep=5pt,                  
    showspaces=false,                
    showstringspaces=false,
    showtabs=false,                  
    tabsize=2,
    language=C
}

\lstdefinestyle{TextStyle}{
    backgroundcolor=\color{white},   
%     stringstyle=\color{mPurple},
    basicstyle=\footnotesize,
    breakatwhitespace=false,         
    breaklines=true,                 
    captionpos=t,                    
    keepspaces=true,
    frame=trlb,
    numbers=left,                    
    numbersep=5pt,                  
    showspaces=false,                
    showstringspaces=false,
    showtabs=false,                  
    tabsize=2,
}

\lstdefinestyle{PythonStyle}{ %
  language = python,
  backgroundcolor=\color{white},   % choose the background color
  basicstyle=\footnotesize,        % size of fonts used for the code
  frame = trlb,
  numbers=left,
  stepnumber=1,
  showstringspaces=false,
  tabsize=1,
  breaklines=true,                 % automatic line breaking only at whitespace
  captionpos=t,                    % sets the caption-position to bottom
  commentstyle=\color{mygreen},    % comment style
  %escapeinside={\%*}{*)},          % if you want to add LaTeX within your code
  keywordstyle=\color{blue},       % keyword style
  stringstyle=\color{mymauve},     % string literal style
}

\lstdefinestyle{BashStyle}{ %
  language = bash,
  backgroundcolor=\color{white},   % choose the background color
  basicstyle=\footnotesize,        % size of fonts used for the code
  frame = trlb,
  numbers=left,
  stepnumber=1,
  showstringspaces=false,
  tabsize=1,
  breaklines=true,                 % automatic line breaking only at whitespace
  captionpos=t,                    % sets the caption-position to bottom
  commentstyle=\color{blue},    % comment style
  %escapeinside={\%*}{*)},          % if you want to add LaTeX within your code
  keywordstyle=\color{orange},       % keyword style
  stringstyle=\color{mymauve},     % string literal style
}
%%###########################

\begin{document}
\begin{center}
\Large{\textbf{ECE 637: Lab 4}}

Rahul Deshmukh

\today
\end{center}

%#########################################
\subsection*{Section 1 Report} 

\begin{enumerate}
\item Print out of kids.tif
\begin{figure}[!hp]
 \centering
 \includegraphics[width=\textwidth]{../code/python/output/section1/kids_gray}
 \caption{kids.tif as gray scale image}
\end{figure}

\clearpage
\item Print out of histogram of kids.tif
\begin{figure}[!hp]
 \centering
 \includegraphics[width=\textwidth]{../code/python/output/section1/kids_histogram}
 \caption{Histogram of kids.tif}
\end{figure}

\clearpage
\item Print out of race.tif
\begin{figure}[!hp]
 \centering
 \includegraphics[width=\textwidth]{../code/python/output/section1/race_gray}
 \caption{race.tif as gray scale image}
\end{figure}

\clearpage
\item Print out of histogram of race.tif
\begin{figure}[!hp]
 \centering
 \includegraphics[width=\textwidth]{../code/python/output/section1/race_histogram}
 \caption{Histogram of race.tif}
\end{figure}

\item For python code refer to \reflst{lst:hist}.
\end{enumerate}

%
\clearpage
\vspace{2ex}
%
\subsection*{Section 2 Report} 
\begin{enumerate}
\item For python code refer to \reflst{lst:hist_eq}.
 
\item Plot of $\hat{F}_x(i)$ for \textit{kids.tif}
\begin{figure}[!hp]
 \centering
 \includegraphics[width=\textwidth]{../code/python/output/section2/kids_cumsum_norm}
 \caption{$\hat{F}_x(i)$ for \textit{kids.tif}}
\end{figure}

\clearpage
\item Plot of equalized image's histogram
\begin{figure}[!hp]
 \centering
 \includegraphics[width=\textwidth]{../code/python/output/section2/kids_hist_eq_histogram}
 \caption{equalized image's histogram for \textit{kids.tif}}
\end{figure}

\clearpage
\item Equalized image
\begin{figure}[!hp]
 \centering
 \includegraphics[width=\textwidth]{../code/python/output/section2/kids_hist_eq}
 \caption{equalized image's histogram for \textit{kids.tif}}
\end{figure}

\end{enumerate}

%
\clearpage
\vspace{2ex}
%
\subsection*{Section 3 Report} 
\begin{enumerate}
\item For python code refer to \reflst{lst:hist_eq}.
 
\item Plot of transformed image and its histogram for \textit{kids.tif} with $T1=80,\quad T2=160$
\begin{figure}[!hp]
 \centering
 \includegraphics[width=\textwidth]{../code/python/output/section3/kids_cont_st_gray}
 \caption{Contrast stretched image for \textit{kids.tif}}
\end{figure}
\begin{figure}[!hp]
 \centering
 \includegraphics[width=\textwidth]{../code/python/output/section3/kids_cont_st_histogram}
 \caption{Histogram of contrast stretched image for \textit{kids.tif}}
\end{figure}
\end{enumerate}

%
\clearpage
\vspace{2ex}
%
\subsection*{Section 4.2 Report} 
\begin{enumerate}
\item Image corresponding to the matching gray level
\begin{figure}[!hp]
 \centering
 \includegraphics[width=\textwidth]{../code/python/output/section4/Array_pattern_gamma_1_710}
 \caption{Array pattern corresponding to matching gray level}
\end{figure}

\item Derivation of the expression which relates the matching gray level to the value of $\gamma$\\

For matching gray level we have the following equation:
\begin{align}
 I_c &= I_g\nonumber\\
 \Rightarrow \frac{I_{255}}{2} &= I_{255}(\frac{g}{255})^{\gamma}\nonumber\\
 \Rightarrow g &= \frac{255}{\gamma}log(0.5)\label{eq:1}
\end{align}

Using \refEq{eq:1} we can calculte the matching gray level.

\item The measured gray level was $g=170$ and measured $\gamma=1.709511$. (For more details refer to the python code at \reflst{lst:gamma_monitor} and bash script at \reflst{lst:bash} with corresponding output log at \reflst{lst:gamma_monitor_log})

\end{enumerate}

%
\clearpage
\vspace{2ex}
%
\subsection*{Section 4.3.1 Report} 
\begin{enumerate}
\item Original and corrected images \textit{linear.tif}
\begin{figure}[!hp]
 \centering
\begin{subfigure}{0.45\textwidth}
 \centering
 \includegraphics[width=\textwidth]{../code/python/output/section4/linear}
 \caption{Original image}
\end{subfigure}
\begin{subfigure}{0.45\textwidth}
 \centering
 \includegraphics[width=\textwidth]{../code/python/output/section4/linear_gamma_corrected}
 \caption{Corrected image with $\gamma=1.709511$}
\end{subfigure} 
 \caption{plots for linear.tif}
\end{figure}

\item Formula used to transform the original image\\

 For CRT we have the following transformation:\vspace{-2ex}
 \begin{align}
 y(x) &= 255(\frac{x}{255})^{\gamma} \label{eq:crt}
 \end{align}
 We want to find correction function $f(x)$ such that $y(f(x))=x$\vspace{-2ex}
 \begin{align}
 y(x) &= 255(\frac{f(x)}{255})^{\gamma}=x\nonumber\\
 \Rightarrow \gamma log(\frac{f(x)}{255}) &= log(\frac{x}{255})\nonumber\\
 \Rightarrow f(x) &= 255e^{\frac{1}{\gamma}log(\frac{x}{255})} \label{eq:linear_correction}
\end{align}
Using \refEq{eq:linear_correction} we can transform the original image for gamma correction.

\end{enumerate}

%
\clearpage
\vspace{2ex}
%
\subsection*{Section 4.3.2 Report} 
\begin{enumerate}
\item Original and corrected images for \textit{gamma15.tif}
\begin{figure}[!hp]
 \centering
 \begin{subfigure}{0.45\textwidth}
 \centering
 \includegraphics[width=\textwidth]{../code/python/output/section4/gamma15}
 \caption{Original image with $\gamma=1.5$}
\end{subfigure}
\begin{subfigure}{0.45\textwidth}
 \centering
 \includegraphics[width=\textwidth]{../code/python/output/section4/gamma15_gamma_corrected}
 \caption{Corrected image with $\gamma=1.709511$}
\end{subfigure}  
 \caption{plots for gamma15.tif}
\end{figure}

\item Procedure used to change gamma correction of the original image, for each pixel value do the following:

\begin{description}
 \item [(i)] Convert $\gamma_1$ image (x) to linear image (y) using \refEq{eq:crt}
 \item [(ii)] Do gamma correction with $\gamma_2$ using \refEq{eq:linear_correction}
\end{description}


For implementation details refer to \reflst{lst:gamma_correction}

\end{enumerate}


\newpage
\subsection*{Source Code}%
\lstinputlisting[caption={Python code for histogram equalization of image},style={PythonStyle},label={lst:hist_eq}]{../code/python/hist_eq.py}
\lstinputlisting[caption={Python code for contrast stretching of image},style={PythonStyle},label={lst:contrast}]{../code/python/contrast_stretch.py}

\newpage
\subsection*{Appendix}
Got to \href{https://github.com/rahuldeshmukh43/Courses/tree/master/ECE637-DigitalImageProcessing-I/labs/lab4/code}{git repo} for complete code.
%python code
\lstinputlisting[caption={Python code for printing histogram of image},style={PythonStyle},label={lst:hist}]{../code/python/histogram.py}
\lstinputlisting[caption={Python code for calculating gamma of display},style={PythonStyle},label={lst:gamma_monitor}]{../code/python/gamma_monitor_4.2.py}
\lstinputlisting[caption={Python code for gamma correction of image},style={PythonStyle},label={lst:gamma_correction}]{../code/python/gamma_correction.py}
%bash code
\lstinputlisting[caption={Bash code for running python code for lab4},style={BashStyle},label={lst:bash}]{../code/python/run.bash}
% terminal output logs
\lstinputlisting[caption={output log for task for section4.2},style={TextStyle},label={lst:gamma_monitor_log}]{../code/python/Array_pattern_gamma.log}

\end{document}

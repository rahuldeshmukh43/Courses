\documentclass[a4paper,11pt]{article}
\usepackage[utf8]{inputenc}
% ------
% LAYOUT
% ------
\textwidth 165mm %
\textheight 230mm %
\oddsidemargin 0mm %
\evensidemargin 0mm %
\topmargin -15mm %
\parindent= 10mm
\setlength{\parskip}{0.5em}

\usepackage[dvips]{graphics}
\usepackage[table]{xcolor}
\usepackage{listings}
\usepackage{color}
\usepackage{graphicx}
% \usepackage{subfigure}
\usepackage{amssymb}
\usepackage{amsfonts}
\usepackage{amsthm}
\usepackage{amsmath}
\usepackage{mathtools}
\usepackage{pifont}

% 
\usepackage[framed,numbered,autolinebreaks,useliterate]{mcode}

\graphicspath{{../pix/}} % put all your figures here.

\definecolor{mygreen}{rgb}{0,0.6,0}
\definecolor{mygray}{rgb}{0.5,0.5,0.5}
\definecolor{mymauve}{rgb}{0.58,0,0.82}

% My short functions
\newcommand{\V}[1]{\boldsymbol{#1}}
\newcommand{\floor}[1]{\lfloor #1 \rfloor}
\newcommand{\mat}[1]{\begin{bmatrix}#1\end{bmatrix}}
\newcommand{\refEq}[1]{Eq. \ref{#1}}
\newcommand{\reflst}[1]{Listing \ref{#1} at page \pageref{#1}}
\newcommand{\reftbl}[1]{Table \ref{#1}}
\newcommand{\reffig}[1]{Figure \ref{#1}}
\newcommand{\dx}[2]{\frac{\partial #1}{\partial #2}}
\newcommand{\ddx}[2]{\frac{\partial^2 #1}{\partial #2^2}}
\newcommand{\cmark}{{\color{blue}\text{\ding{51}}}}%
\newcommand{\xmark}{{\color{red}\text{\ding{55}}}}%
\newcommand{\TODO}{{\color{red}TODO}}
%%###########################

\begin{document}
\begin{center}
\Large{\textbf{ECE 580: Homework 5}}

Rahul Deshmukh

\today
\end{center}

%#########################################
\subsection*{Exercise 1} 
In my Canonical GA, I am using the following parameter settings:
\begin{itemize}
 \item Number of bits used to represent each variable: 10 (resolution=0.0098)
 \item Population size: 40
 \item Number of iterations: 30
 \item Probability for cross-over: 0.9
 \item Probability of Mutation: 0.01
 \item Selection Method: tournament selection method-2
\end{itemize}

\noindent After carrying out several trials, I obtain an optimal function value of $1.7928e-05$ for the optimal solution as
$x^* = \mat{-0.0049&   -0.0049 }^T$. The plot for best, average, and the worst objective function values in the population for every generation is at \reffig{fig:ga_canon_conv}.

\begin{figure}[!h]
 \centering
 \includegraphics[width=0.6\textwidth]{ga_canon_conv}
 \caption{Plot of Average, Best and Worse function values for GA}
 \label{fig:ga_canon_conv}
\end{figure}

\noindent For main file for GA refer to \reflst{lst:cga_main}. The fitness function can be refered at \reflst{lst:cga_fit} \&  \reflst{lst:cga_fit2}. The encoding and decoding functions can be found at \reflst{lst:cga_encode} \& \reflst{lst:cga_decode} respectively. The function for Tournament selection is at \reflst{lst:tournament_selection}. The function for crossover, mutation and elitism can be refered at \reflst{lst:cga_xover}, \reflst{lst:cga_mut} \& \reflst{lst:ga_elitism} respectively.

%
\clearpage
\vspace{2ex}
%
\subsection*{Exercise 2} 
In my Real-Number GA, I am using the following parameter settings:
\begin{itemize}
 \item Population size: 40
 \item Number of iterations: 30
 \item Probability for cross-over: 0.9
 \item Crossover-Method: Convex Combination
 \item Probability of Mutation: 0.01
 \item Selection Method: tournament selection method-2
\end{itemize}

\noindent After carrying out several trials, I obtain an optimal function value of $1.7333e-11$ for the optimal solution as
$x^* = 1.0e-05*\mat{0.0195&    0.8318}^T$. The plot for best, average, and the worst objective function values in the population for every generation is at \reffig{fig:ga_real_conv}.

\begin{figure}[!h]
 \centering
 \includegraphics[width=0.6\textwidth]{ga_real_conv}
 \caption{Plot of Average, Best and Worse function values for GA}
 \label{fig:ga_real_conv}
\end{figure}

\noindent For main file for GA refer to \reflst{lst:rga_main}. The fitness function can be refered at \reflst{lst:cga_fit} \&  \reflst{lst:cga_fit2}. The function for Tournament selection is at \reflst{lst:tournament_selection}. The function for crossover, mutation and elitism can be refered at \reflst{lst:rga_xover}, \reflst{lst:rga_mut} \& \reflst{lst:ga_elitism} respectively.

%
\clearpage
\vspace{2ex}
%
\subsection*{Exercise 3}
The given LP problem is:
\begin{align*}
 \text{maximize} \quad& -4x_1 -3x_2\\
 \text{subject to} \quad& 5x_1 +x_2 \geq 11\\
 & 2x_1 +x_2 \geq 8\\
 & x_1 + 2x_2 \geq 7\\
 & x_1,x_2 \geq 0
\end{align*}

\noindent We first convert the above problem to standard form:
\begin{align*}
 \text{minimize} \quad& 4x_1 +3x_2\\
 \text{subject to} \quad& 5x_1 +x_2 - x_3=  11\\
 & 2x_1 +x_2 -x_4 = 8\\
 & x_1 + 2x_2 -x_5 = 7\\
 & x_1,x_2,x_3,x_4,x_5 \geq 0
\end{align*}

\noindent We then solve the above problem using Two-phase simplex method. The computations (with pivot elements in boxes) are as follows:
\begin{align}
\text{Phase 1:}&\nonumber\\
%
\begin{bmatrix}
 \V{A}& \V{I}& \V{b}\\
 \V{0}^T& \V{1}^T& 0
\end{bmatrix}
% 
&=\begin{bmatrix}
5&	1&	-1&	0&	0&	1&	0&	0&	11\\
2&	1&	0&	-1&	0&	0&	1&	0&	8\\
1&	2&	0&	0&	-1&	0&	0&	1&	7\\
0&	0&	0&	0&	0&	1&	1&	1&	0
\end{bmatrix}\nonumber\\
% 
&=\begin{bmatrix}
\fbox{{\color{red}5}}&	1&	-1&	0&	0&	1&	0&	0&	11\\
2&	1&	0&	-1&	0&	0&	1&	0&	8\\
1&	2&	0&	0&	-1&	0&	0&	1&	7\\
-8&	-4&	1&	1&	1&	0&	0&	0&	-26
\end{bmatrix}\nonumber\\
% 
&=\begin{bmatrix}
1&	1/5&	-1/5&	0&	0&	1/5&	0&	0&	11/5\\
0&	3/5&	2/5&	-1&	0&	-2/5&	1&	0&	18/5\\
0&	\fbox{{\color{red}9/5}}&	1/5&	0&	-1&	-1/5&	0&	1&	24/5\\
0&	-12/5&	-3/5&	1&	1&	8/5&	0&	0&	-42/5
\end{bmatrix}\nonumber\\
% 
&=\begin{bmatrix}
1&	0&	-2/9&	0&	1/9&	2/9&	0&	-1/9&	5/3\\
0&	0&	\fbox{{\color{red}1/3}}&	-1&	1/3&	-1/3&	1&	-1/3&	2\\
0&	1&	1/9&	0&	-5/9&	-1/9&	0&	5/9&	8/3\\
0&	0&	-1/3&	1&	-1/3&	4/3&	0&	4/3&	-2
\end{bmatrix}\nonumber\\
% 
&=\begin{bmatrix}
1&	0&	0&	-2/3&	1/3&	0&	2/3&	-1/3&	3\\
0&	0&	1&	-3&	1&	-1&	3&	-1&	6\\
0&	1&	0&	1/3&	-2/3&	0&	-1/3&	2/3&	2\\
0&	0&	0&	0&	0&	1&	1&	1&	0
\end{bmatrix}\nonumber
\end{align}
%
\begin{align}
\text{Phase 2:}&\nonumber\\
%
\begin{bmatrix}
 \V{A}& \V{b}\\
 \V{c}^T& 0
\end{bmatrix}
%
&=\begin{bmatrix}
1&	0&	0&	-2/3&	1/3&	3\\
0&	0&	1&	-3&	1&	6\\
0&	1&	0&	1/3&	-2/3&	2\\
4&	3&	0&	0&	0&	0
\end{bmatrix}\nonumber\\
% 
&=\begin{bmatrix}
1&	0&	0&	-2/3&	1/3&	3\\
0&	0&	1&	-3&	1&	6\\
0&	1&	0&	1/3&	-2/3&	2\\
0&	0&	0&	5/3&	2/3&	-18
\end{bmatrix}\nonumber
\end{align}

\noindent The optimal solution is given by $x_1^*=3, x_2^*=2$ with maximum function value as $-4x_1^* -3x_2^* = -18$

\noindent For MATLAB function for this problem refer to \reflst{lst:lp_2phsimplex} \& \reflst{lst:lp_simplex} and the call to the function can be referred at \reflst{lst:lp_main} with corresponding output at \reflst{lst:lp_op}.
%
\subsection*{Exercise 4} 
The dual problem is given by:
\begin{align*}
 \text{maximize} \quad& \V{\lambda}^T\V{b}\\
 \text{subject to} \quad& \V{\lambda}^T\V{A} \leq c^T
\end{align*}
\noindent that is:
\begin{align*}
 \text{maximize} \quad& 11\lambda_1 +8\lambda_2 + 7\lambda_3\\
 \text{subject to} \quad& \mat{\lambda_1 &\lambda_2& \lambda_3}^T
 \mat{5& 1\\
 2& 1\\
 1& 2} 
 \leq \mat{4& 3}\\
 & \lambda_1,\lambda_2,\lambda_3 \geq 0
\end{align*}
Since we have already obtained the optimal BFS $\V{x}^* = \mat{3& 2& 0& 0& 0}^T$ corresponding to the optimal basis 
$\V{B} = \mat{5& 1& -1\\ 2& 1& 0\\ 1& 2& 0}$ and cost coefficients $\V{c}_{\V{B}} = \mat{4& 3& 0}^T$. From theorem of duality, we have
\begin{align*}
 \V{\lambda}^T\V{b} &= \V{c}^T\V{x} = \V{c}_{\V{B}}^T\V{B}^{-1}\V{b}\\
 \Rightarrow \V{\lambda}^T &= \V{c}_{\V{B}}^T\V{B}^{-1}\\
 &= \mat{4& 3& 0}^T \mat{5& 1& -1\\ 2& 1& 0\\ 1& 2& 0}^{-1}\\
 &= \frac{1}{-3}\mat{4& 3& 0}^T\mat{0& -2& 1\\0& 1& -2\\ 3& -9& 3}^T\\
 &\V{\lambda}^{* T}= \mat{0& 5/3& 2/3}\\
 &\V{\lambda}^{* T}\V{b} = 18
\end{align*}
% 
% 
\clearpage
\vspace{2ex}
\subsection*{Exercise 5} 
We have to solve the following constrained optimization problem:
\begin{align*}
 \text{maximize} \quad& f(x) = 4x_1 + x_2^2\\
 \text{eqv minimize} \quad& -f(x)= -4x_1 - x_2^2\\
 \text{subject to} \quad& h_1(x) = x_1^2 + x_2^2 -9 = 0\\\\
 \text{lagrangian}\quad& l(\V{x},\lambda_1) =  -4x_1 - x_2^2 + \lambda_1(x_1^2 + x_2^2 -9)
\end{align*}
Using FONC:
\begin{align*}
\nabla_{\V{x}} l(\V{x},\lambda_1) &= 
\mat{-4& -2x_2} + \lambda_1\mat{2x_1& 2x_2} = \V{0}^T\\
\Rightarrow& \mat{-4+2\lambda_1 x_1\\ 2x_2(\lambda_1-1)} = \mat{0\\0}\\
\nabla_{\lambda_1} l(\V{x},\lambda_1) &= 
x_1^2 + x_2^2 -9 = 0\\
\end{align*}
The solution to the above set of equations is:
\begin{align*}
 \V{x}^*_a = \mat{x_1\\x_2\\ \lambda_1} = \mat{\pm3\\ 0\\ \pm\frac{2}{3}} ,\quad
 \V{x}^*_b =\mat{x_1\\x_2\\ \lambda_1} = \mat{2\\ \pm \sqrt{5}\\ 1}
\end{align*}
Computing the Hessian of lagrangian:
\begin{align*}
 L(\V{x},\lambda_1) &= 2\mat{\lambda_1& 0\\0& (\lambda_1-1)}\\
 \Rightarrow L(\V{x}^*_b, \lambda^*_1=  1)&=2\mat{1& 0\\0& 0}\succcurlyeq0 \Rightarrow \V{x}^*_b\text{ is a possible minimizer of  }-f(\V{x})\\
 L(\V{x}^*_{a2}, \lambda^*_1 = -\frac{2}{3})&= 2\mat{-\frac{2}{3}& 0\\0& -\frac{5}{3}}\preccurlyeq0 \Rightarrow \V{x}^*_b\text{ is a possible maximizer of }-f(\V{x})\\
\end{align*}
For SONC we need to find the Tangent Space $T(\V{x}^*)$:
\begin{align*}
 T(\V{x}^*_b) & = \{y: \nabla_{\V{x}}h_1^T(x)y = 0   \}\\
 & = \{y: \mat{2x_1^*& 2x_2^*}y = 0   \}\\
 &=  \{y: \mat{2& \pm\sqrt{5}}y = 0   \}\\
& =  \{y:  y=t\mat{1\\ \mp\frac{\sqrt{5}}{2}} ,\quad t\in\mathbb{R} -\{0\}  \}\\
y^TL(\V{x}^*_b)y &= t^2 \mat{1& \mp\frac{\sqrt{5}}{2}} 2\mat{1& 0\\0& 0} \mat{1\\ \mp\frac{\sqrt{5}}{2}}\\
&= 2t^2> 0 \quad \forall \quad t\in\mathbb{R}-\{0\}
\end{align*}
Therefore $\V{x}^*_b =\mat{2\\ \pm \sqrt{5}\\ 1}$ is a strict minimizer of $-f(x)$ and thus is a maximizer of $f(x)$

\clearpage
\vspace{2ex}
\subsection*{Exercise 6} 
We have to solve the following optimization problem:
\begin{align*}
	\text{maximize} \quad& r(x) = \frac{18x_1^2 - 8x_1x_2 +12x_2^2}{2x_1^2 + 2x_2^2}
\end{align*}
Suppose the above problem has a solution $x^*$ which maximizes the function. Then $x = tx^*, t\in \mathbb{R}$ can be another possible solution. Therefore we convert the above problem to a constrained maximization problem with unique solution by enforcing the denominator to be 1. The converted problem is given by :
\begin{align*}
 \text{maximize} \quad& f(x) = 18x_1^2 - 8x_1x_2 +12x_2^2\\
	\text{subject to} \quad& h(x)= 1 - (2x_1^2 + 2x_2^2) = 0
\end{align*}
Converting to minimization problem:
\begin{align*}
 \text{minimize} \quad& -f(x) = -\frac{1}{2}\V{x}^T\mat{36& -8\\-8& 24}\V{x}\\
 \text{subject to} \quad& h(x)= 1 - \frac{1}{2}\V{x}^T\mat{4&0\\0&4}\V{x}\\
 \text{lagrangian}\quad& l(x,\lambda) = -\frac{1}{2}\V{x}^T\mat{36& -8\\-8& 24}\V{x} + \lambda(1 - \frac{1}{2}\V{x}^T\mat{4&0\\0&4}\V{x})
\end{align*}
FONC:
\begin{align*}
 \nabla_{\lambda} l(\V{x},\lambda)&= 2x_1^2 + 2x_2^2-1 = 0\\
 \nabla_{\V{x}} l(\V{x},\lambda) &= 
 \mat{-36& 8\\8& -24}\V{x} - \lambda(\mat{4&0\\0&4}\V{x})=\V{0}\\
 \Rightarrow& \mat{4&0\\0&4}^{-1}\mat{-36& +8\\+8& -24}\V{x} = \lambda\V{x}\\
 \Rightarrow& \mat{-9& +2\\+2& -6}\V{x} = \lambda\V{x}\\
 \Rightarrow& det(\mat{-9& +2\\+2& -6}-\lambda \V{I}_2) = 0
 \Rightarrow \lambda = -10,-5\\ 
 & \V{x} = \alpha\mat{-2\\1}, \beta\mat{1\\2}\\
\end{align*}
The solution to the above equations is:
\begin{align*}
 \V{x}^*_a = \mat{x_1\\x_2\\\lambda}=\mat{\mp\frac{2}{\sqrt{10}} \\ \pm\frac{1}{\sqrt{10}}\\-10},\quad 
 \V{x}^*_b =\mat{x_1\\x_2\\\lambda}=\mat{\pm\frac{1}{\sqrt{10}} \\ \pm\frac{2}{\sqrt{10}}\\-5}
\end{align*}
SONC:
\begin{align*}
 L(\V{x},\lambda) &=\mat{-36& 8\\8& -24} -\lambda\mat{4&0\\0&4}\\
 & = (\mat{-9-\lambda& +2\\+2& -6-\lambda})\mat{4&0\\0&4}\\
 \Rightarrow L(\V{x},\lambda=-10) &= \mat{4& 8\\8& 16} \succcurlyeq 0 \text{ is a possible minimizer of } -f(x)\\
 \Rightarrow L(\V{x},\lambda=-5) &= \mat{-16& +8\\+8& -4}\preccurlyeq 0 \text{ is a possible maximizer of } -f(x)
\end{align*}
SOSC: We need to find the Tangent Space $T(x^*)$
\begin{align*}
T(\V{x}^*_a) & = \{y: \nabla_{\V{x}}h_1^T(x)y = 0   \}\\
& = \{y: \mat{4&0\\0&4}\mat{\mp\frac{2}{\sqrt{10}} \\ \pm\frac{1}{\sqrt{10}}}y = 0   \}\\
&=\{y: y = t\mat{1\\2},\quad t\in\mathbb{R}-\{0\}\}\\
\Rightarrow y^TL(\V{x}^*_a)y &= 100t^2 > 0 \quad \forall \quad t\in\mathbb{R}-\{0\}
\end{align*}
Therefore the solution corresponding to $\lambda=-10$, $\V{x}^*_a =\mat{\mp\frac{2}{\sqrt{10}} \\ \pm\frac{1}{\sqrt{10}}}$ is a strict minimizer of $-f(x)$ and thus is a maximizer of $f(x)$.\\ 

\noindent Therefore all possible maximizers of $$r(x) = \frac{18x_1^2 - 8x_1x_2 +12x_2^2}{2x_1^2 + 2x_2^2}$$
are of the form $\V{x}^* =t\mat{\mp\frac{2}{\sqrt{10}} \\ \pm\frac{1}{\sqrt{10}}} \quad \forall \quad t \in \mathbb{R}$

\clearpage
\vspace{2ex}
\subsection*{Exercise 7}
We need to find the extremizers for problems (a)-(c)
\subsubsection*{Problem (a)}
The given problem is:
\begin{align*}
 f(\V{x}) &= x_1^2 + x_2^2 -2x_1 -10x_2 +26\\
 \text{subject to}\quad g(\V{x})&= \mat{\frac{x_2}{5}-x_1^2\\ 5x_1 +\frac{x_2}{2}-5} \leq \V{0}
\end{align*}
The gradients are given by:
\begin{align*}
 \nabla f(x) &= \mat{2x_1 -2\\ 2x_2-10}\\
 \nabla g(x) &= \mat{-2x_1 & 1/5\\ 5&  1/2}\\
 F(x) &= \mat{2&0\\0&0}\\
 G_1(x) &= \mat{-2 & 0\\ 0& 0}\\
 G_2(x) &= \mat{0 & 0\\ 0 &0}
\end{align*}
The lagrangian is given by:
\begin{align*}
 l(x,\lambda,\mu) &= f(x) + \lambda^Th(x) + \mu^Tg(x)\\
 \nabla_x l(x,\lambda,\mu) &= \mat{2x_1 -2\\ 2x_2-10} +
 \mat{-2x_1 & 1/5\\ 5& 1/2}^T\mat{\mu_1\\ \mu_2}\\
 L(x,\lambda,\mu)&= F(x) + \sum_{i=1}^{m}\lambda_i H_i(x) + \sum_{i=1}^{p}\mu_i G_i(x)\\
 &=  \mat{2&0\\0&0} +\mu_1\mat{-2 & 0\\ 0& 0} + \mu_2\mat{0 & 0\\ 0 &0}\\
 &= \mat{2&0\\0&0} +\mu_1\mat{-2 & 0\\ 0& 0}
\end{align*}
FONC: Using KKT:-\\

\noindent\underline{Case 1:} $\mu_1=0, \mu_2=0$
\begin{align*}
 \nabla_x l(x,\lambda,\mu) &= \mat{2x_1 -2\\ 2x_2-10}=0\\
 \text{subject to}\quad g_1(x)&=\frac{x_2}{5}-x_1^2 \leq 0\\
 g_2(x)&=5x_1 +\frac{x_2}{2}-5\leq 0
\end{align*}
The solution to the above system of equations is:
\begin{align*}
 \V{x}^* &= \mat{1\\5}\\
 g_1(x)&= 0 \leq 0 \quad\cmark\\
 g_2(x)&= 5+5/2 \leq 0 \quad\xmark
\end{align*}
Therefore the above solution is infeasible\\

\noindent\underline{Case 2:} $\mu_1=0, \mu_2>0$
\begin{align*}
 \nabla_x l(x,\lambda,\mu) &= \mat{2x_1 -2\\ 2x_2-10} +\mat{-2x_1 & 1/5\\ 5& 1/2}^T\mat{0\\ \mu_2} =0\\
 \text{subject to}\quad g_1(x)&=\frac{x_2}{5}-x_1^2 \leq 0\\
 g_2(x)&=5x_1 +\frac{x_2}{2}-5= 0
\end{align*}
The solution to the above system of equations is:
\begin{align*}
 \V{x}^* &= \mat{x_1\\x_2} = \mat{51/101\\ 500/101}\\
 \mu_2^*& = 20/101\\
 g_1(x)&=\frac{x_2}{5}-x_1^2 = (100*101-51^2)/101^2 \leq 0 \quad \xmark\\
\end{align*}
There is no feasible solution.

\noindent\underline{Case 3:} $\mu_1>0, \mu_2>0$ 
\begin{align*}
 \nabla_x l(x,\lambda,\mu) &= \mat{2x_1 -2\\ 2x_2-10} +\mat{-2x_1 & 1/5\\ 5& 1/2}^T\mat{\mu_1\\ \mu_2} =0\\
 \text{subject to}\quad g_1(x)&=\frac{x_2}{5}-x_1^2 = 0\\
 g_2(x)&=5x_1 +\frac{x_2}{2}-5= 0
\end{align*}
The solution to the above system of equations is:
\begin{align*}
 \V{x}^* &= \mat{x_1\\x_2} = \mat{-1\pm\sqrt{3}\\ 20\mp\sqrt{3}}\\
 \mu_1^* &= \frac{304-202\sqrt{3}}{-2\sqrt{3}} \quad \cmark\\
 \mu_2^* &= (-300 + 200\sqrt{3} -2\mu_1)/5 >0 \quad\cmark
\end{align*}

\noindent SONC:
\begin{align*}
 L(\V{x}^*,\mu^*)& = \mat{2&0\\0&0} +\mu_1\mat{-2 & 0\\ 0& 0}\\
 T(\V{x}^*)&= \{y: \mat{-2x_1 & 1/5\\ 5&  1/2}y = 0 \}\\
 &= \{y: \mat{-2(-1\pm\sqrt{3})& 1/5\\ 5&  1/2}y = 0 \}\\
 &= \{y: y=\V{0}\}\\
 \Rightarrow& y^TL(\V{x}^*,\mu^*)y = 0 \geq 0
\end{align*}
Thereofore $\V{x}^*  = \mat{-1\pm\sqrt{3}\\ 20\mp\sqrt{3}}$ is an extremizer.

\noindent SOSC:
\begin{align*}
 \tilde{T}(\V{x}^*)&= \{y: \mat{-2x_1 & 1/5\\ 5&  1/2}y = 0 \}\\
 &= \{y: \mat{-2(-1\pm\sqrt{3})& 1/5\\ 5&  1/2}y = 0 \}\\
 &= \{y: y=\V{0}\}\\
 \Rightarrow& y^TL(\V{x}^*,\mu^*)y = 0 > 0 \quad \xmark
\end{align*}
As $\V{x}^*$ does not satisfy SONC, it is not a minimizer. It is a saddle point.


\noindent\underline{Case 4:} $\mu_1>0,\mu_2=0$ 
\begin{align*}
 \nabla_x l(x,\lambda,\mu) &= \mat{2x_1 -2\\ 2x_2-10} +\mat{-2x_1 & 1/5\\ 5& 1/2}^T\mat{\mu_1\\ 0} =0\\
 \text{subject to}\quad g_1(x)&=\frac{x_2}{5}-x_1^2 = 0\\
 g_2(x)&=5x_1 +\frac{x_2}{2}-5\leq 0
\end{align*}
The solution to the above system of equations is:
\begin{align*}
 \V{x}^* &= \mat{x_1\\x_2}=\mat{-48+10\sqrt{23}\\23020-4800\sqrt{23}}\\
 \mu_1^*&= \frac{52+10\sqrt{23}}{4} >0 \quad \cmark\\
 g_2(x)&= 5x_1 +\frac{x_2}{2}-5 = 11265 - 2350\sqrt{23} \leq 0 \quad\cmark
\end{align*}
Therefore $\V{x}^*= \mat{-48+10\sqrt{23}\\23020-4800\sqrt{23}}$ is a possible extremizer as it satisfies the KKT conditions.

\noindent SONC:
\begin{align*}
 L(\V{x}^*,\mu^*)& = \mat{2&0\\0&0} +\mu_1\mat{-2 & 0\\ 0& 0}\\
 T(\V{x}^*)& = \{y: \mat{-2x_1 & 1/5\\ 5&  1/2}y = 0\}\\
 &= \{y: \mat{-2(-48+10\sqrt{23}) & 1/5\\ 5&  1/2}y = 0\}\\
 & = \{y: y=\V{0}\}\\
 \Rightarrow& y^TL(\V{x}^*,\mu^*)y = 0 \geq 0
\end{align*}
Therefore $\V{x}^*$ is an extremizer as it satisfies SONC.

\noindent SOSC:
\begin{align*}
 \tilde{T}(\V{x}^*)& = \{y: \mat{-2x_1& 1/5}y=0\}\\
 & = \{y: y=t\mat{1\\10x_1} \quad \forall \quad t\in \mathbb{R}-\{0\}\}\\
 \Rightarrow& y^TL(\V{x}^*,\mu^*)y = 50x_1(1+x_1)t^2\\
 & = 50*(-48+10\sqrt{23})(1-48+10\sqrt{23})t^2 < 0 \quad \forall \quad t\in \mathbb{R}-\{0\}\}
\end{align*}
Therefore $\V{x}^*=\mat{-48+10\sqrt{23}\\23020-4800\sqrt{23}}$ is a strict maximizer. 

\clearpage
\subsubsection*{Problem (b)}
\begin{align*}
 f(\V{x}) &= x_1^2 +x_2^2\\
 \text{subject to}\quad g(\V{x})&= \mat{-x_1\\-x_2\\-x_1-x_2+5} \leq \V{0}
\end{align*}
The gradients are given by:
\begin{align*}
 \nabla f(x) &= \mat{2x_1\\ 2x_2}\\
 \nabla g(x) &= \mat{-1& 0\\ 0& -1\\-1& -1}\\
 F(x) &= \mat{2&0\\0&2}\\
 G_1(x) &= \mat{0 & 0\\ 0& 0}\\
 G_2(x) &= \mat{0 & 0\\ 0 &0}\\
 G_3(x) &= \mat{0 & 0\\ 0 &0}
\end{align*}
The lagrangian is given by:
\begin{align*}
 l(x,\lambda,\mu) &= f(x) + \lambda^Th(x) + \mu^Tg(x)\\
 \nabla_x l(x,\lambda,\mu) &= \mat{2x_1\\ 2x_2} +
\mat{-1& 0\\ 0& -1\\-1& -1}^T\mat{\mu_1\\ \mu_2\\ \mu_3}\\
L(x,\lambda,\mu)&= F(x) + \sum_{i=1}^{m}\lambda_i H_i(x) + \sum_{i=1}^{p}\mu_i G_i(x)\\
&= \mat{2&0\\0&2} + \mu_1\mat{0 & 0\\ 0& 0} + \mu_2\mat{0 & 0\\ 0& 0} + \mu_3\mat{0 & 0\\ 0& 0}\\
&= \mat{2&0\\0&2} \succ 0 \quad \text{in all of }\mathbb{R}^2
\end{align*}
Thus as $L(x,\lambda,\mu)$ is PD in all of $\mathbb{R}^2$, the SONC is always satisfied for any solution satisfying KKT.

\noindent FONC: Using KKT:-\\

\noindent\underline{Case 1:} $\mu_1>0, \mu_2>0, \mu_3>0$
\begin{align*}
 \nabla_x l(x,\lambda,\mu) &= \mat{2x_1\\ 2x_2} +
\mat{-1& 0\\ 0& -1\\-1& -1}^T\mat{\mu_1\\ \mu_2\\ \mu_3}\\
% 
 \text{subject to}\quad g_1(x)&=-x_1 = 0\\
 g_2(x)&=-x_2 = 0\\
 g_3(x)&= -x_1 -x_2 +5=0
\end{align*}
There is no feasible solution to the above system of equations.

\noindent\underline{Case 2:} $\mu_1>0, \mu_2>0, \mu_3=0$
\begin{align*}
 \nabla_x l(x,\lambda,\mu) &= \mat{2x_1\\ 2x_2} +
\mat{-1& 0\\ 0& -1\\-1& -1}^T\mat{\mu_1\\ \mu_2\\ 0}\\
% 
 \text{subject to}\quad g_1(x)&=-x_1 = 0\\
 g_2(x)&=-x_2 = 0\\
 g_3(x)&= -x_1 -x_2 +5 = 5\leq0 \quad \xmark
\end{align*}
There is no feasible solution to the above system of equations.

\noindent\underline{Case 3:} $\mu_1>0, \mu_2=0, \mu_3>0$
\begin{align*}
 \nabla_x l(x,\lambda,\mu) &= \mat{2x_1\\ 2x_2} +
\mat{-1& 0\\ 0& -1\\-1& -1}^T\mat{\mu_1\\ 0\\ \mu_3}\\
% 
 \text{subject to}\quad g_1(x)&=-x_1 = 0\\
 g_2(x)&=-x_2 \leq 0\\
 g_3(x)&= -x_1 -x_2 +5 =0 
\end{align*}
The solution to the above set of equations is :
\begin{align*}
 \V{x}^*&=\mat{x_1\\x_2} = \mat{0\\5}\\
 \mu_3^*&=10\\
 \mu_1^*&=-10>0 \quad \xmark
\end{align*}
There is no feasible solution.

\noindent\underline{Case 4:} $\mu_1=0, \mu_2>0, \mu_3>0$
\begin{align*}
 \nabla_x l(x,\lambda,\mu) &= \mat{2x_1\\ 2x_2} +
\mat{-1& 0\\ 0& -1\\-1& -1}^T\mat{0\\ \mu_2\\ \mu_3}\\
% 
 \text{subject to}\quad g_1(x)&=-x_1 \leq 0\\
 g_2(x)&=-x_2 = 0\\
 g_3(x)&= -x_1 -x_2 +5 =0 
\end{align*}
The solution to the above set of equations is :
\begin{align*}
 \V{x}^*&=\mat{x_1\\x_2} = \mat{5\\0}\\
 \mu_3^*&= 10\\
 \mu_2^*&= -10>0 \quad \xmark
\end{align*}
There is no feasible solution.

\noindent\underline{Case 5:} $\mu_1>0, \mu_2=0, \mu_3=0$
\begin{align*}
 \nabla_x l(x,\lambda,\mu) &= \mat{2x_1\\ 2x_2} +
\mat{-1& 0\\ 0& -1\\-1& -1}^T\mat{\mu_1\\ 0\\ 0}\\
% 
 \text{subject to}\quad g_1(x)&=-x_1 = 0\\
 g_2(x)&=-x_2 \leq 0\\
 g_3(x)&= -x_1 -x_2 +5 \leq0 
\end{align*}
The solution to the above set of equations is :
\begin{align*}
 \V{x}^*&=\mat{x_1\\x_2} = \mat{0\\0}\\
 g_2(x)&=-x_2 =0\leq 0 \quad \cmark\\
 g_3(x)&= -x_1 -x_2 +5 = 5 \leq0 \quad \xmark\\
 \mu_1^*&= 0>0 \quad \xmark
\end{align*}
There is no feasible solution.

\noindent\underline{Case 6:} $\mu_1=0, \mu_2>0, \mu_3=0$
\begin{align*}
 \nabla_x l(x,\lambda,\mu) &= \mat{2x_1\\ 2x_2} +
\mat{-1& 0\\ 0& -1\\-1& -1}^T\mat{0\\ \mu_2\\ 0}\\
% 
 \text{subject to}\quad g_1(x)&=-x_1 \leq 0\\
 g_2(x)&=-x_2 = 0\\
 g_3(x)&= -x_1 -x_2 +5 \leq0 
\end{align*}
The solution to the above set of equations is :
\begin{align*}
 \V{x}^*&=\mat{x_1\\x_2} = \mat{0\\0}\\
 g_1(x)&=-x_1 =0\leq 0 \quad \cmark\\
 g_3(x)&= -x_1 -x_2 +5 = 5 \leq0 \quad \xmark\\
 \mu_2^*&= 0>0 \quad \xmark
\end{align*}
There is no feasible solution.

\noindent\underline{Case 7:} $\mu_1=0, \mu_2=0, \mu_3=0$
\begin{align*}
 \nabla_x l(x,\lambda,\mu) &= \mat{2x_1\\ 2x_2} +
\mat{-1& 0\\ 0& -1\\-1& -1}^T\mat{0\\ 0\\ 0}\\
% 
 \text{subject to}\quad g_1(x)&=-x_1 \leq 0\\
 g_2(x)&=-x_2 \leq 0\\
 g_3(x)&= -x_1 -x_2 +5 \leq0 
\end{align*}
The solution to the above set of equations is :
\begin{align*}
 \V{x}^*&=\mat{x_1\\x_2} = \mat{0\\0}\\
 g_1(x)&=-x_1 =0\leq 0 \quad \cmark\\
 g_2(x)&=-x_2 =0\leq 0 \quad \cmark\\
 g_3(x)&= -x_1 -x_2 +5 = 5 \leq0 \quad \xmark
\end{align*}
There is no feasible solution.

\noindent\underline{Case 8:} $\mu_1=0, \mu_2=0, \mu_3>0$
\begin{align*}
 \nabla_x l(x,\lambda,\mu) &= \mat{2x_1\\ 2x_2} +
\mat{-1& 0\\ 0& -1\\-1& -1}^T\mat{0\\ 0\\ \mu_3}\\
% 
 \text{subject to}\quad g_1(x)&=-x_1 \leq 0\\
 g_2(x)&=-x_2 \leq 0\\
 g_3(x)&= -x_1 -x_2 +5 =0 
\end{align*}
The solution to the above set of equations is :
\begin{align*}
 \V{x}^*&=\mat{x_1\\x_2} = \mat{5/2\\5/2},  \mu_3^*= 5\\
 g_1(x)&=-x_1 =-5/2\leq 0 \quad \cmark\\
 g_2(x)&=-x_2 =-5/2\leq 0 \quad \cmark
\end{align*}
Therefore $\V{x}^*=\mat{x_1\\x_2} = \mat{5/2\\5/2}$ is a possible extremizer. Also since $L(x,\lambda,\mu)\succ0$ for all of $\mathbb{R}^2$, thus $\V{x}^*$ a strict minimizer.

\clearpage
\subsubsection*{Problem (c)}
\begin{align*}
 f(\V{x}) &= x_1^2+6x_1x_2-4x_1 -2x_2\\
 \text{subject to}\quad g(\V{x})&= \mat{x_1^2 +2x_2-1\\ 2x_1-2x_2-1} \leq \V{0}
\end{align*}
The gradients are given by:
\begin{align*}
 \nabla f(x) &= \mat{2x_1 +6x_2 -4\\ 6x_1 -2}\\
 \nabla g(x) &= \mat{2x_1 & 2\\ 2&  -2}\\
 F(x) &= \mat{2&6\\6&0}\\
 G_1(x) &= \mat{1 & 0\\ 0& 0}\\
 G_2(x) &= \mat{0 & 0\\ 0 &0}
\end{align*}
The lagrangian is given by:
\begin{align*}
 l(x,\lambda,\mu) &= f(x) + \lambda^Th(x) + \mu^Tg(x)\\
 \nabla_x l(x,\lambda,\mu) &=  \mat{2x_1 +6x_2 -4\\ 6x_1 -2} +
\mat{2x_1 & 2\\ 2&  -2}^T\mat{\mu_1\\ \mu_2}\\
L(x,\lambda,\mu)&= F(x) + \sum_{i=1}^{m}\lambda_i H_i(x) + \sum_{i=1}^{p}\mu_i G_i(x)\\
&= \mat{2&6\\6&0} + \mu_1\mat{1 & 0\\ 0& 0} + \mu_2\mat{0 & 0\\ 0& 0}\\
&= \mat{2&6\\6&0} + \mu_1\mat{1 & 0\\ 0& 0}
\end{align*}

\noindent FONC: Using KKT:-

\noindent\underline{Case 1:} $\mu_1=0, \mu_2=0$
\begin{align*}
 \nabla_x l(x,\lambda,\mu) &= \mat{2x_1 +6x_2 -4\\ 6x_1 -2} +
\mat{2x_1 & 2\\ 2&  -2}^T\mat{0\\ 0}\\
% 
 \text{subject to}\quad g_1(x)&=x_1^2 +2x_2-1 \leq 0\\
 g_2(x)&=2x_1-2x_2-1\leq 0
\end{align*}
The solution to the above set of equations is :
\begin{align*}
 \V{x}^*&=\mat{x_1\\x_2} = \mat{1/3\\5/9}\\
 g_1(x)&=x_1^2 +2x_2-1 =2/9\leq 0 \quad \xmark\\
 g_2(x)&=2x_1-2x_2-1 =-13/9\leq 0 \quad \cmark
\end{align*}
There is no feasible solution.

\noindent\underline{Case 2:} $\mu_1>0,\mu_2=0$
\begin{align*}
 \nabla_x l(x,\lambda,\mu) &= \mat{2x_1 +6x_2 -4\\ 6x_1 -2} +
\mat{2x_1 & 2\\ 2&  -2}^T\mat{\mu_1\\ 0}\\
% 
 \text{subject to}\quad g_1(x)&=x_1^2 +2x_2-1 = 0\\
 g_2(x)&=2x_1-2x_2-1\leq 0
\end{align*}
The solution to the above set of equations gives:
\begin{align*}
x_1 &= \frac{1-\mu_1}{3}\\
x_2 &= \frac{2}{3} - \frac{1}{9}(1-\mu_1^2)\\
g_1(x)&= x_1^2 +2x_2 -1 = 0\\
\Rightarrow& 3\mu_1^2 -2\mu_1+ 2 = 0\\
\Rightarrow \mu_1^*&= \frac{2\pm\sqrt{4-24}}{2} \notin \mathbb{R}  
\end{align*}
There is no feasible solution.

\noindent\underline{Case 3:} $\mu_1>0, \mu_2>0$
\begin{align*}
 \nabla_x l(x,\lambda,\mu) &= \mat{2x_1 +6x_2 -4\\ 6x_1 -2} +
\mat{2x_1 & 2\\ 2&  -2}^T\mat{\mu_1\\ \mu_2}\\
% 
 \text{subject to}\quad g_1(x)&=x_1^2 +2x_2-1 = 0\\
 g_2(x)&=2x_1-2x_2-1= 0
\end{align*}
The solution to the above set of equations gives :
\begin{align*}
 \V{x}^*&=\mat{x_1\\x_2} = \mat{-1\pm \sqrt{3}\\-3/2 \pm \sqrt{3}}\\
 \mu_1^*&=\pm\frac{1}{\sqrt{3}}(23\mp14\sqrt{3}) >0 \quad \xmark\\
  \mu_2^*&= \mu_1 + (-4\pm3\sqrt{3})
\end{align*}
There is no feasible solution.

\noindent\underline{Case 4:} $\mu_1=0, \mu_2>0$
\begin{align*}
 \nabla_x l(x,\lambda,\mu) &= \mat{2x_1 +6x_2 -4\\ 6x_1 -2} +
\mat{2x_1 & 2\\ 2&  -2}^T\mat{0\\ \mu_2}\\
% 
 \text{subject to}\quad g_1(x)&=x_1^2 +2x_2-1 \leq 0\\
 g_2(x)&=2x_1-2x_2-1= 0
\end{align*}
The solution to the above set of equations is :
\begin{align*}
 \V{x}^*&=\mat{x_1\\x_2} = \mat{9/14\\1/7}\\
 \mu_2^*&=13/14 >0 \quad \cmark\\
 g_1(x)&=x_1^2 +2x_2-1 = -59/196\leq 0 \quad \cmark\\
\end{align*}
Therefore $\V{x}^*=\mat{9/14\\1/7}$ is a possible extremizer. 

\noindent SONC: 
\begin{align*}
 L(\V{x}^*,\mu^*) &= \mat{2&6\\6&0}\\
 T(\V{x}^*) &=\{y: \mat{2x_1& 2\\2& -2}y = \V{0}\}\\
 &=\{y: \mat{9/7& 2\\2& -2}y = \V{0}\}\\
 &= \{y: y=\V{0}\}\\
 \Rightarrow y^TL(\V{x}^*,\mu^*)y = 0 \geq 0
\end{align*}
Thus $\V{x}^*$ satisfies SONC and therefore is an extremizer. 

\noindent SOSC:
\begin{align*}
 \tilde{T}(\V{x}^*) &= \{y:\mat{2&-2}y = 0\}\\
 &= \{y:y=t\mat{1\\1} \quad \forall \quad t\in \mathbb{R} -\{0\}\}\\
 \Rightarrow y^TL(\V{x}^*,\mu^*)y &= 14t^2>0 \quad \forall \quad t\in \mathbb{R} -\{0\}
\end{align*}
Therefore $\V{x}^*=\mat{9/14\\1/7}$ is a strict minimizer.




\newpage
\subsection*{MATLAB Code}
%
\subsection*{Canonical GA Code}
\lstinputlisting[caption={Canonical GA Main Code},label={lst:cga_main}]{../GA_griewank.m}
\lstinputlisting[caption={Fitness function},label={lst:cga_fit}]{../fitness_griewank.m}
\lstinputlisting[caption={Griewank function},label={lst:cga_fit2}]{../griewank_fun.m}
\lstinputlisting[caption={Encoding function},label={lst:cga_encode}]{../../OptimModule/optimizers/global/GA/encode.m}
\lstinputlisting[caption={Decoding function},label={lst:cga_decode}]{../../OptimModule/optimizers/global/GA/decode.m}
\lstinputlisting[caption={Roulette-wheel selection function},label={lst:ga_selection_roulette}]{../../OptimModule/optimizers/global/GA/roulette.m}
\lstinputlisting[caption={Tournament selection function},label={lst:tournament_selection}]{../../OptimModule/optimizers/global/GA/tournament_selection.m}
\lstinputlisting[caption={Cross-over function},label={lst:cga_xover}]{../../OptimModule/optimizers/global/GA/two_point_crossover.m}
\lstinputlisting[caption={Mutation function},label={lst:cga_mut}]{../../OptimModule/optimizers/global/GA/mutation.m}
\lstinputlisting[caption={Elitism function},label={lst:ga_elitism}]{../../OptimModule/optimizers/global/GA/elitism.m}
\lstinputlisting[caption={Logging function},label={lst:ga_logger}]{../../OptimModule/optimizers/global/GA/log_f.m}
% 
\subsection*{Real GA Code}
\lstinputlisting[caption={Real GA Main Code},label={lst:rga_main}]{../Real_GA_griewank.m}
\lstinputlisting[caption={Cross-over function},label={lst:rga_xover}]{../../OptimModule/optimizers/global/GA/Real_Num_GA/crossover.m}
\lstinputlisting[caption={Mutation function},label={lst:rga_mut}]{../../OptimModule/optimizers/global/GA/Real_Num_GA/mutation.m}
% 
\subsection*{Linear programming Code}
\lstinputlisting[caption={Linprog Main Code},label={lst:lp_main}]{../linprog_call.m}
\lstinputlisting[caption={Two Phase Simplex},label={lst:lp_2phsimplex}]{../../OptimModule/optimizers/linprog/mylinprog.m}
\lstinputlisting[caption={Simplex Method},label={lst:lp_simplex}]{../../OptimModule/optimizers/linprog/simplex.m}
\begin{lstlisting}[caption={LP output}, label={lst:lp_op}]
 ** Optimum Solution found using mylinprog **

x_str =

       3       
       2       


fval =

      18    
\end{lstlisting}
% 
\end{document}

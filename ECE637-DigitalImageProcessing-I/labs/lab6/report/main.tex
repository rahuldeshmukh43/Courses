\documentclass[a4paper,11pt]{article}

% ------
% LAYOUT
% ------
\textwidth 165mm %
\textheight 230mm %
\oddsidemargin 0mm %
\evensidemargin 0mm %
\topmargin -15mm %
\parindent= 10mm
\setlength{\parskip}{0.5em}

\usepackage[utf8]{inputenc}
\usepackage[T1]{fontenc}
\usepackage[dvips]{graphics}
\usepackage[table]{xcolor}
\usepackage{listings}
\usepackage{color}
\usepackage{graphicx}
% \usepackage{subfigure}
\usepackage{amssymb}
\usepackage{amsfonts}
\usepackage{amsthm}
\usepackage{amsmath}
\usepackage{mathtools}
\usepackage{pifont}
\usepackage[outdir=./]{epstopdf}
\usepackage{caption}
\usepackage{subcaption}
\usepackage{hyperref}
\usepackage{textcomp}
% 
% \usepackage[framed,numbered,autolinebreaks,useliterate]{mcode}
\hypersetup{
    colorlinks=true,
    linkcolor=blue,
    filecolor=magenta,      
    urlcolor=cyan,
}

\urlstyle{same}

% My short functions
\newcommand{\V}[1]{\boldsymbol{#1}}
\newcommand{\floor}[1]{\lfloor #1 \rfloor}
\newcommand{\mat}[1]{\begin{bmatrix}#1\end{bmatrix}}
\newcommand{\refEq}[1]{Eq. \ref{#1}}
\newcommand{\reflst}[1]{Listing \ref{#1} at page \pageref{#1}}
\newcommand{\reftbl}[1]{Table \ref{#1}}
\newcommand{\reffig}[1]{Figure \ref{#1}}
\newcommand{\dx}[2]{\frac{\partial #1}{\partial #2}}
\newcommand{\ddx}[2]{\frac{\partial^2 #1}{\partial #2^2}}
\newcommand{\cmark}{{\color{blue}\text{\ding{51}}}}%
\newcommand{\xmark}{{\color{red}\text{\ding{55}}}}%
\newcommand{\TODO}{{\color{red}TODO}}

%listing styles
\definecolor{mGreen}{rgb}{0,0.6,0}
\definecolor{mGray}{rgb}{0.5,0.5,0.5}
\definecolor{mPurple}{rgb}{0.58,0,0.82}
\definecolor{mygreen}{rgb}{0,0.6,0}
\definecolor{mygray}{rgb}{0.5,0.5,0.5}
\definecolor{mymauve}{rgb}{0.58,0,0.82}

\lstset{upquote=true}

\lstdefinestyle{CStyle}{
    backgroundcolor=\color{white},   
    commentstyle=\color{mGreen},
    keywordstyle=\color{magenta},
    stringstyle=\color{mPurple},
    basicstyle=\footnotesize,
    breakatwhitespace=false,         
    breaklines=true,                 
    captionpos=t,                    
    keepspaces=true,
    frame=trlb,
    numbers=left,                    
    numbersep=5pt,                  
    showspaces=false,                
    showstringspaces=false,
    showtabs=false,                  
    tabsize=2,
    language=C
}

\lstdefinestyle{TextStyle}{
    backgroundcolor=\color{white},   
%     stringstyle=\color{mPurple},
    basicstyle=\footnotesize,
    breakatwhitespace=false,         
    breaklines=true,                 
    captionpos=t,                    
    keepspaces=true,
    frame=trlb,
    numbers=left,                    
    numbersep=5pt,                  
    showspaces=false,                
    showstringspaces=false,
    showtabs=false,                  
    tabsize=2,
}

\lstdefinestyle{PythonStyle}{ %
  language = python,
  backgroundcolor=\color{white},   % choose the background color
  basicstyle=\footnotesize,        % size of fonts used for the code
  frame = trlb,
  numbers=left,
  stepnumber=1,
  showstringspaces=false,
  tabsize=1,
  breaklines=true,                 % automatic line breaking only at whitespace
  captionpos=t,                    % sets the caption-position to bottom
  commentstyle=\color{mygreen},    % comment style
  %escapeinside={\%*}{*)},          % if you want to add LaTeX within your code
  keywordstyle=\color{blue},       % keyword style
  stringstyle=\color{mymauve},     % string literal style
}

\lstdefinestyle{BashStyle}{ %
  language = bash,
  backgroundcolor=\color{white},   % choose the background color
  basicstyle=\footnotesize,        % size of fonts used for the code
  frame = trlb,
  numbers=left,
  stepnumber=1,
  showstringspaces=false,
  tabsize=1,
  breaklines=true,                 % automatic line breaking only at whitespace
  captionpos=t,                    % sets the caption-position to bottom
  commentstyle=\color{blue},    % comment style
  %escapeinside={\%*}{*)},          % if you want to add LaTeX within your code
  keywordstyle=\color{orange},       % keyword style
  stringstyle=\color{mymauve},     % string literal style
}
%%###########################

\begin{document}
\begin{center}
\Large{\textbf{ECE 637: Lab 6}}

Rahul Deshmukh\\\href{mailto:deshmuk5@purdue.edu}{{\color{blue}deshmuk5@purdue.edu}}

\today
\end{center}

%#########################################
\subsection*{Section 2 Report} 
\begin{enumerate}
\item Plot of $x_0(\lambda),y_0(\lambda)$ and $z_0(\lambda)$:
 \begin{figure}[!hp]
  \centering
  \includegraphics[width=\textwidth]{../code/python/output/ex2/xyz}
  \caption{Plot of $x_0(\lambda),y_0(\lambda)$ and $z_0(\lambda)$}
 \end{figure}

\clearpage
\item Plot of $l_0(\lambda),m_0(\lambda)$ and $s_0(\lambda)$
\begin{figure}[!hp]
  \centering
  \includegraphics[width=\textwidth]{../code/python/output/ex2/lms}
  \caption{Plot of $l_0(\lambda),m_0(\lambda)$ and $s_0(\lambda)$}
 \end{figure}

\clearpage
\item Plot of $D_{65}$ and fluorescent illuminants
\begin{figure}[!hp]
  \centering
  \includegraphics[width=\textwidth]{../code/python/output/ex2/illums_spectrum}
  \caption{Plot of $D_{65}$ and fluorescent illuminants}
  \label{fig:illums}
 \end{figure}
\end{enumerate}
For python code refer to \reflst{lst:ex2}.

\clearpage
\subsection*{Section 3 Report} 
Chromaticity diagram:
\begin{figure}[!hp]
  \centering
  \includegraphics[width=\textwidth]{../code/python/output/ex3/chrom_diag}
  \caption{Chromaticity diagram}
 \end{figure}
For python code refer to \reflst{lst:ex3}.
%  \lstinputlisting[caption={output log for section2},style={TextStyle},label={lst:sec2log}]{../code/python/output/section_2/sec2.log}

%
\clearpage
\vspace{2ex}
\subsection*{Section 4 Report} 
\begin{enumerate}
 \item The Matrix $M_{709\_D65}$:
 \lstinputlisting[caption={output log showing $M_{709\_D65}$},style={TextStyle},label={lst:ex4_log}]{../code/python/output/ex4/d65.log}
 
 \item Images obtained from $D_{65}$ and fluorescent light sources:
 \begin{figure}[!hp]
 \centering
 \begin{subfigure}{0.45\textwidth}
%  \centering
 \includegraphics[width=\textwidth]{../code/python/output/ex4/rgb_d65_gamma_corrected}
 \caption{$D_{65}$ light source image}
 \end{subfigure}
%  
 \begin{subfigure}{0.45\textwidth}
%  \centering
 \includegraphics[width=\textwidth]{../code/python/output/ex4/rgb_ee_gamma_corrected}
 \caption{fluorescent light source image}
 \end{subfigure}
\end{figure} 

 \item Qualitative description of the difference between the two images: Based on visual comparison we can see that the fluorescent image has a stronger yellow tint. This can be explained using \reffig{fig:illums} where fluorescent illumination response crosses $D_{65}$ close to $\lambda=550$ which is the yellow color ($\lambda=580$). Fluorescent illuminant also cuts out the voilet color($\lambda=450$) compared to $D_{65}$.

\end{enumerate}
For python code refer to \reflst{lst:ex4} and \reflst{lst:gamma_correction}

\clearpage
\vspace{2ex}
\subsection*{Section 5 Report} 
\begin{figure}[!hp]
  \centering
  \includegraphics[width=\textwidth]{../code/python/output/ex5/chrom_plot}
  \caption{Color chromaticity diagram}
 \end{figure}

For python code refer to \reflst{lst:ex5}

\clearpage
\subsection*{Appendix}
Got to \href{https://github.com/rahuldeshmukh43/Courses/tree/master/ECE637-DigitalImageProcessing-I/labs/lab6/code/python/}{git repo} for complete code.
%python code
\lstinputlisting[caption={Python code for section 2},style={PythonStyle},label={lst:ex2}]{../code/python/ex2.py}
\lstinputlisting[caption={Python code for section 3},style={PythonStyle},label={lst:ex3}]{../code/python/ex3.py}
\lstinputlisting[caption={Python code for section 4},style={PythonStyle},label={lst:ex4}]{../code/python/ex4.py}
\lstinputlisting[caption={Python code for section 5},style={PythonStyle},label={lst:ex5}]{../code/python/ex5.py}
\lstinputlisting[caption={Python code for gamma correction},style={PythonStyle},label={lst:gamma_correction}]{../code/python/gamma_correction.py}
%bash code
\lstinputlisting[caption={Bash code for running python code},style={BashStyle},label={lst:bash}]{../code/python/run.bash}

\end{document}

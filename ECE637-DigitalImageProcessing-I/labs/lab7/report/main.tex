\documentclass[a4paper,11pt]{article}

% ------
% LAYOUT
% ------
\textwidth 165mm %
\textheight 230mm %
\oddsidemargin 0mm %
\evensidemargin 0mm %
\topmargin -15mm %
\parindent= 10mm
\setlength{\parskip}{0.5em}

\usepackage[utf8]{inputenc}
\usepackage[T1]{fontenc}
\usepackage[dvips]{graphics}
\usepackage[table]{xcolor}
\usepackage{listings}
\usepackage{color}
\usepackage{graphicx}
% \usepackage{subfigure}
\usepackage{amssymb}
\usepackage{amsfonts}
\usepackage{amsthm}
\usepackage{amsmath}
\usepackage{mathtools}
\usepackage{pifont}
\usepackage[outdir=./]{epstopdf}
\usepackage{caption}
\usepackage{subcaption}
\usepackage{hyperref}
\usepackage{textcomp}
% 
% \usepackage[framed,numbered,autolinebreaks,useliterate]{mcode}
\hypersetup{
    colorlinks=true,
    linkcolor=blue,
    filecolor=magenta,      
    urlcolor=cyan,
}

\urlstyle{same}

% My short functions
\newcommand{\V}[1]{\boldsymbol{#1}}
\newcommand{\floor}[1]{\lfloor #1 \rfloor}
\newcommand{\mat}[1]{\begin{bmatrix}#1\end{bmatrix}}
\newcommand{\refEq}[1]{Eq. \ref{#1}}
\newcommand{\reflst}[1]{Listing \ref{#1} at page \pageref{#1}}
\newcommand{\reftbl}[1]{Table \ref{#1}}
\newcommand{\reffig}[1]{Figure \ref{#1}}
\newcommand{\dx}[2]{\frac{\partial #1}{\partial #2}}
\newcommand{\ddx}[2]{\frac{\partial^2 #1}{\partial #2^2}}
\newcommand{\cmark}{{\color{blue}\text{\ding{51}}}}%
\newcommand{\xmark}{{\color{red}\text{\ding{55}}}}%
\newcommand{\TODO}{{\color{red}TODO}}

%listing styles
\definecolor{mGreen}{rgb}{0,0.6,0}
\definecolor{mGray}{rgb}{0.5,0.5,0.5}
\definecolor{mPurple}{rgb}{0.58,0,0.82}
\definecolor{mygreen}{rgb}{0,0.6,0}
\definecolor{mygray}{rgb}{0.5,0.5,0.5}
\definecolor{mymauve}{rgb}{0.58,0,0.82}

\lstset{upquote=true}

\lstdefinestyle{CStyle}{
    backgroundcolor=\color{white},   
    commentstyle=\color{mGreen},
    keywordstyle=\color{magenta},
    stringstyle=\color{mPurple},
    basicstyle=\footnotesize,
    breakatwhitespace=false,         
    breaklines=true,                 
    captionpos=t,                    
    keepspaces=true,
    frame=trlb,
    numbers=left,                    
    numbersep=5pt,                  
    showspaces=false,                
    showstringspaces=false,
    showtabs=false,                  
    tabsize=2,
    language=C
}

\lstdefinestyle{TextStyle}{
    backgroundcolor=\color{white},   
%     stringstyle=\color{mPurple},
    basicstyle=\footnotesize,
    breakatwhitespace=false,         
    breaklines=true,                 
    captionpos=t,                    
    keepspaces=true,
    frame=trlb,
    numbers=left,                    
    numbersep=5pt,                  
    showspaces=false,                
    showstringspaces=false,
    showtabs=false,                  
    tabsize=2,
}

\lstdefinestyle{PythonStyle}{ %
  language = python,
  backgroundcolor=\color{white},   % choose the background color
  basicstyle=\footnotesize,        % size of fonts used for the code
  frame = trlb,
  numbers=left,
  stepnumber=1,
  showstringspaces=false,
  tabsize=1,
  breaklines=true,                 % automatic line breaking only at whitespace
  captionpos=t,                    % sets the caption-position to bottom
  commentstyle=\color{mygreen},    % comment style
  %escapeinside={\%*}{*)},          % if you want to add LaTeX within your code
  keywordstyle=\color{blue},       % keyword style
  stringstyle=\color{mymauve},     % string literal style
}

\lstdefinestyle{BashStyle}{ %
  language = bash,
  backgroundcolor=\color{white},   % choose the background color
  basicstyle=\footnotesize,        % size of fonts used for the code
  frame = trlb,
  numbers=left,
  stepnumber=1,
  showstringspaces=false,
  tabsize=1,
  breaklines=true,                 % automatic line breaking only at whitespace
  captionpos=t,                    % sets the caption-position to bottom
  commentstyle=\color{blue},    % comment style
  %escapeinside={\%*}{*)},          % if you want to add LaTeX within your code
  keywordstyle=\color{orange},       % keyword style
  stringstyle=\color{mymauve},     % string literal style
}
%%###########################

\begin{document}
\begin{center}
\Large{\textbf{ECE 637: Lab 7}}

Rahul Deshmukh\\\href{mailto:deshmuk5@purdue.edu}{{\color{blue}deshmuk5@purdue.edu}}

\today
\end{center}

%#########################################
\subsection*{Section 1 Report} 
\begin{enumerate}
\item Plot of four original images:
 \begin{figure}[!hp]
  \centering
\begin{subfigure}{0.45\textwidth}
%  \centering
  \includegraphics[width=\textwidth]{../pix/img14g}
 \caption{Plot of img14g.tif}
 \end{subfigure}
\begin{subfigure}{0.45\textwidth}
%  \centering
  \includegraphics[width=\textwidth]{../pix/img14bl}
 \caption{Plot of img14bl.tif}
 \end{subfigure}
 \vfill
 \begin{subfigure}{0.45\textwidth}
%  \centering
  \includegraphics[width=\textwidth]{../pix/img14gn}
 \caption{Plot of img14gn.tif}
 \end{subfigure}
 \begin{subfigure}{0.45\textwidth}
%  \centering
  \includegraphics[width=\textwidth]{../pix/img14sp}
 \caption{Plot of img14sp.tif}
 \end{subfigure}  
 \end{figure}

\clearpage
\item Output of filtering of blurred and noisy image
 \begin{figure}[!hp]
  \centering
\begin{subfigure}{0.45\textwidth}
%  \centering
  \includegraphics[width=\textwidth]{../code/python/output/sec1/img14bl_predicted}
 \caption{Plot of restored img14bl.tif}
 \end{subfigure}
 \vfill
 \begin{subfigure}{0.45\textwidth}
%  \centering
  \includegraphics[width=\textwidth]{../code/python/output/sec1/img14gn_predicted}
 \caption{Plot of restored img14gn.tif}
 \end{subfigure}
 \begin{subfigure}{0.45\textwidth}
%  \centering
  \includegraphics[width=\textwidth]{../code/python/output/sec1/img14sp_predicted}
 \caption{Plot of restored img14sp.tif}
 \end{subfigure}  
 \end{figure}

\clearpage
\item Computed MSME filter log files:
\lstinputlisting[caption={output log for img14bl},style={TextStyle},label={lst:sec1log1}]{../code/python/output/sec1/bl.log}
\lstinputlisting[caption={output log for img14gn},style={TextStyle},label={lst:sec1log2}]{../code/python/output/sec1/gn.log}
\lstinputlisting[caption={output log for img14sp},style={TextStyle},label={lst:sec1log3}]{../code/python/output/sec1/sp.log}
\end{enumerate}

For python code refer to \reflst{lst:sec1} and \reflst{lst:bash1}.

\clearpage
\subsection*{Section 2 Report} 
Results of median filtering:

\begin{figure}[!hp]
  \centering
 \begin{subfigure}{0.45\textwidth}
%  \centering
  \includegraphics[width=\textwidth]{../code/C-code/run/output/img14gn_medianfilter}
 \caption{median filtered img14gn.tif}
 \end{subfigure}
 \begin{subfigure}{0.45\textwidth}
%  \centering
  \includegraphics[width=\textwidth]{../code/C-code/run/output/img14sp_medianfilter}
 \caption{median filtered img14sp.tif}
 \end{subfigure}  
 \end{figure}
 
For C-code refer to \reflst{lst:sec2} and \reflst{lst:bash2}.


\clearpage
\subsection*{Appendix}
Got to \href{https://github.com/rahuldeshmukh43/Courses/tree/master/ECE637-DigitalImageProcessing-I/labs/lab7/code/}{git repo} for complete code.
%python code
\lstinputlisting[caption={Python code for section 1},style={PythonStyle},label={lst:sec1}]{../code/python/section1.py}
%C-code
\lstinputlisting[caption={C-code for section 2},style={CStyle},label={lst:sec2}]{../code/C-code/src/MedianFiltering.c}
%bash code
\lstinputlisting[caption={Bash code for running python code for section 1},style={BashStyle},label={lst:bash1}]{../code/python/run.sh}
\lstinputlisting[caption={Bash code for running C-code for section 2},style={BashStyle},label={lst:bash2}]{../code/C-code/run/run.sh}

\end{document}

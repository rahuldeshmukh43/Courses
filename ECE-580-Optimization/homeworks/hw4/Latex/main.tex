\documentclass[11pt]{article}

% ------
% LAYOUT
% ------
\textwidth 165mm %
\textheight 230mm %
\oddsidemargin 0mm %
\evensidemargin 0mm %
\topmargin -15mm %
\parindent= 10mm

\usepackage[dvips]{graphics}
% \usepackage{multirow,multicol}
\usepackage[table]{xcolor}
\usepackage{listings}
\usepackage{color}
\usepackage{graphicx}
% \usepackage{subfigure}
\usepackage{amssymb}
\usepackage{amsfonts}
\usepackage{amsthm}
\usepackage{amsmath}
% \usepackage{physics}
\usepackage{enumerate}
\usepackage{mathtools}
% \usepackage{cancel}
% \usepackage{epstopdf}

\setlength{\parskip}{0.5em}

\graphicspath{{../pix/}} % put all your figures here.
\usepackage[framed,numbered,autolinebreaks,useliterate]{mcode}

\definecolor{mygreen}{rgb}{0,0.6,0}
\definecolor{mygray}{rgb}{0.5,0.5,0.5}
\definecolor{mymauve}{rgb}{0.58,0,0.82}
% \lstset{ %
%   language = python
%   backgroundcolor=\color{white},   % choose the background color
%   basicstyle=\footnotesize,        % size of fonts used for the code
%   frame = trlb,
%   numbers=left,
%   stepnumber=1,
%   showstringspaces=false,
%   tabsize=1,
%   breaklines=true,                 % automatic line breaking only at whitespace
%   captionpos=t,                    % sets the caption-position to bottom
%   commentstyle=\color{mygreen},    % comment style
%   %escapeinside={\%*}{*)},          % if you want to add LaTeX within your code
%   keywordstyle=\color{blue},       % keyword style
%   stringstyle=\color{mymauve},     % string literal style
% }

% % Define Language
% \lstdefinelanguage{Output}
% {
%   % list of keywords
% %   morekeywords={
% %     import,
% %     if,
% %     while,
% %     for
% %   },
%   sensitive=false, % keywords are not case-sensitive
% %   morecomment=[l]{//}, % l is for line comment
% %   morecomment=[s]{*}{*/}, % s is for start and end delimiter
% %   morestring=[b]" % defines that strings are enclosed in double quotes
% }
% \lstset{ %
%   language = {Output}
% %   backgroundcolor=\color{white},   % choose the background color
% %   basicstyle=\footnotesize,        % size of fonts used for the code
%   frame = trlb,
%   numbers=left,
%   stepnumber=1,
%   showstringspaces=false,
%   tabsize=1,
%   breaklines=true,                 % automatic line breaking only at whitespace
%   captionpos=t,                    % sets the caption-position to top
%   }
%%##########################
% My short functions
\newcommand{\V}[1]{\pmb{#1}}
\newcommand{\rp}{{\color{red}\pmb{+}}}
\newcommand{\bm}{{\color{blue}\pmb{-}}}
\newcommand{\floor}[1]{\lfloor #1 \rfloor}
\newcommand{\mat}[1]{\begin{bmatrix}#1\end{bmatrix}}
\newcommand{\refEq}[1]{Eq. \ref{#1}}
\newcommand{\reflst}[1]{Listing \ref{#1} at page \pageref{#1}}
\newcommand{\reftbl}[1]{Table \ref{#1}}
\newcommand{\reffig}[1]{Figure \ref{#1}}
\newcommand{\dx}[2]{\frac{\partial #1}{\partial #2}}
\newcommand{\ddx}[2]{\frac{\partial^2 #1}{\partial #2^2}}
%%###########################

\begin{document}
\begin{center}
\Large{\textbf{ECE 580: Homework 3}}

Rahul Deshmukh

\today
\end{center}

%#########################################


\subsection*{Exercise 1} 
For this problem, we need to construct two matrices $A_1,A_2$ such that $A_2^{\dagger}A_1^{\dagger} \neq (A_1A_2)^{\dagger} $.

For example lets take 
\begin{align}
 A_1 &= \mat{ 0 & 1\\ 1& 1}= \mat{1\\0}\mat{1& 0}\nonumber\\
 A_2 &= \mat{1 & 0\\ 0& 0}\nonumber\\
 A_1^{\dagger} &= A_1^{-1} = \mat{-1 & 1\\ 1 & 0}\nonumber\\
 A_2^{\dagger} &= (\mat{1\\0}\mat{1& 0})^{\dagger}\nonumber\\
 &= \mat{1\\0}\left( \mat{1&0} \mat{1 & 0\\ 0& 0} \mat{1\\0} \right)^{-1}\mat{1 & 0}\nonumber\\
 & = \mat{1\\0}\left( \mat{1&0} \mat{1\\0}\right)^{-1}\mat{1 & 0}\nonumber\\
 & = \mat{1\\0}\mat{1 & 0}\nonumber\\
 & = \mat{1 & 0\\ 0& 0}\nonumber\\
 A_2^{\dagger}A_1^{\dagger} &= \mat{-1 & 1\\ 1 & 0}\mat{1 & 0\\ 0& 0} \nonumber\\
 \Rightarrow A_2^{\dagger}A_1^{\dagger} & = \mat{-1& 1\\ 0& 0 } \label{eq:1}
\end{align}

Now the RHS is given by:
\begin{align}
 (A_1A_2) &= \mat{ 0 & 1\\ 1& 1}\mat{1 & 0\\ 0& 0} = \mat{0 & 0\\ 1 & 0} = \mat{0 \\1}\mat{1 &0 } \nonumber\\
 (A_1A_2)^{\dagger} &=  (\mat{0 \\1}\mat{1 &0 })^{\dagger} \nonumber\\
 & = \mat{1 \\0} \left( \mat{0 & 1} \mat{0 & 0\\ 1 & 0} \mat{1 \\0}\right)^{-1} \mat{0 &1} \nonumber\\
 & = \mat{1 \\0} \left( \mat{0 & 1} \mat{0 \\1}\right)^{-1} \mat{0 &1} \nonumber\\
 & = \mat{1 \\0} \mat{0 &1} \nonumber\\
 \Rightarrow (A_1A_2)^{\dagger} &= \mat{0 & 1\\ 0 & 0} \label{eq:2}
\end{align}

Therefore from \refEq{eq:1}\& \refEq{eq:2} we can see that $A_2^{\dagger}A_1^{\dagger} \neq (A_1A_2)^{\dagger}$.

%
\newpage
\vspace{2ex}
%
\subsection*{Exercise 2} 

For Exercise 2 \& Exercise 3 we will be working with the Griewank function which is defined by the \reflst{lst:pso_fun}. The function plot over the domain $[-5,5]\text{x}[5,5]$ is at \reffig{fig:pso_fun}.

\begin{figure}[!h]
 \centering
 \includegraphics[width=0.7\textwidth]{surf_plot}
 \caption{Surface plot of Griewank function}
 \label{fig:pso_fun}
\end{figure}

In my particle swarm algorithm, I am using the following parameter settings:
\begin{itemize}
 \item Swarm size: $d=40$
 \item Number of iterations: 100
 \item Inertial constant: $\omega = 0.8$
 \item Cognitive constant: $c_1 = 2$
 \item Social constant: $c_2 = 2$
%  \item Bounds for velocity: $-v_{max}\leq v\leq v_{max}$ with $v_{max}=0.1$
\end{itemize}
After several trials, I get the optimal solution as $\V{x}=\mat{0.0007 & 0.0023}^T$ with a function value 1.6217e-06. It should be noted that the PSO algorithm can get stuck in local minima as it not gauranteed to find the global minima due to finite size of the population and finite number of iterations.

\noindent The location of the optimal solution on contour plot is at \reffig{fig:pso_min_cont}.

\noindent The plot for best, average, and the worst objective function values in the population for every generation is at \reffig{fig:pso_min}

\begin{figure}[!h]
 \centering
 \includegraphics[width=0.7\textwidth]{pso_min_traj}
 \caption{Plot of optimal solution(red X) on contours of objective function}
 \label{fig:pso_min_cont}
\end{figure}

\begin{figure}[!h]
 \centering
 \includegraphics[width=0.7\textwidth]{plot_pso_min}
 \caption{Plot of Average, Best and Worse function values for PSO}
 \label{fig:pso_min}
\end{figure}

\noindent For MATLAB function for this problem refer to \reflst{lst:pso} \& \reflst{lst:pso_fun} and the call to the function can be referred at \reflst{lst:main_code} with corresponding output at \reflst{lst:op}.
%
\clearpage
\vspace{2ex}
%
\subsection*{Exercise 3}
For Maximization problem, we just multiply the Griewank function with negative one and then minimize it with the same parameter settings. After several trials, I get the optimal solution as $\V{x}=\mat{   0.0000 &    4.4474}^T$ with a function value 2.0049. 

\noindent The location of the optimal solution on contour plot is at \reffig{fig:pso_max_cont}.

\noindent The plot for best, average, and the worst objective function values in the population for every generation is at \reffig{fig:pso_max}.

\begin{figure}[!h]
 \centering
 \includegraphics[width=0.7\textwidth]{pso_max_traj}
 \caption{Plot of optimal solution (red X) on contours of objective function}
 \label{fig:pso_max_cont}
\end{figure}

\begin{figure}[!h]
 \centering
 \includegraphics[width=0.7\textwidth]{plot_pso_max}
 \caption{Plot of Average, Best and Worse function values for PSO}
 \label{fig:pso_max}
\end{figure}



\noindent For MATLAB function for this problem refer to \reflst{lst:pso} \& \reflst{lst:pso_fun} and the call to the function can be referred at \reflst{lst:main_code} with corresponding output at \reflst{lst:op}.
%
\clearpage
\vspace{2ex}
\subsection*{Exercise 4} 
In my GA algorithm, I am using the following parameter settings:
\begin{itemize}
 \item Population size: 100
 \item Number of iterations: 200
 \item Probability for cross-over: 0.8
\end{itemize}

\noindent For the TSP, our design variable (\V{x}) is a 10 dimensional vector with each individual component ($x_i$) indicating the city visited at the $i^{th}$ turn. We can have a total of $9!=362880$ possible routes. 

\noindent To obtain the initial population we randomly permute the numbers in the range 1-10 and then proceed with fitness evaulation. We then carry out selection, cross-over, elitism and fitness evaluation repeadtedly till the number of iterations are satisfied. 

\noindent For crossover, we dont want to carry-out an operation which might result in a in-feasible sample. For example, with 10 cities and a resolution of 1 we can represent the decimal number with 4 bits. However, the coded word cannot be binary representations of numbers greater than 9. Therefore to avoid such a problem, we carry out cross-over by just inverting the visiting order between two randomly chosen coordinates of parent-vector.

\noindent For selection, I am using the method-2 of tournament-selection.

\noindent After carrying out several trials, I obtain a shortest route of 27.2133. The order of cities for this route is $\mat{ 9   & 10&     1&     6&     2&     5&     3&     4&     8&     7}$. 

\noindent The shortest route found using GA is at \reffig{fig:ga_route}. The plot for best, average, and the worst objective function values in the population for every generation is at \reffig{fig:ga_conv}.

\begin{figure}[!h]
  \centering
 \includegraphics[width=0.7\textwidth]{ga_best_route}
 \caption{Shortest route calculated using GA}
 \label{fig:ga_route}
\end{figure}

\begin{figure}[!h]
 \centering
 \includegraphics[width=0.7\textwidth]{ga_conv}
 \caption{Plot of Average, Best and Worse function values for GA}
 \label{fig:ga_conv}
\end{figure}
\noindent For main file for GA refer to \reflst{lst:ga_main}. The fitness function can be refered at \reflst{lst:ga_fit}. The encoding and decoding functions can be found at \reflst{lst:ga_encode}\& \reflst{lst:ga_decode} respectively. The function for Tournament selection is at \reflst{lst:tournament_selection}. The function for crossover and elitism can be refered at \reflst{lst:ga_xover} \& \reflst{lst:ga_elitism} respectively.


\clearpage
\vspace{2ex}
\subsection*{Exercise 5} 
For this problem we are required to solve the following problem
\begin{align*}
 \V{x}^* &= \operatorname*{argmax}_{\V{x}} \V{c}^T\V{x}\\
 \text{subject to } &A\V{x} \leq \V{b}\\
 & \V{x}\geq \V{0}\\
 \text{where}\\
 \V{c}^T &= \mat{6 & 4 & 7& 5}\\
 A & = \mat{
 1& 2& 1 & 2\\
 6& 5& 3 & 2\\
 3& 4& 9 & 12\\
 }\\
 \V{b}& = \mat{20\\ 100\\ 75} 
\end{align*}

\noindent We convert the above problem to a minimization problem by multiplyting $\V{c}^T$ by negative one and then solve using MATLAB's \texttt{linprog()} function which works only for a minimization problem.

\noindent We obtain an optimal solution as $\V{x}^* = \mat{15.0 & 0.0& 3.3333& 0.0}^T$ with the maximum function value of 113.3333.

\noindent The MATLAB code for linprog can be found at \reflst{lst:main_code} with corresponding output at \reflst{lst:op}.

\newpage
\subsection*{MATLAB Code}
\lstinputlisting[caption={Main Code}, label={lst:main_code}]{../main.m}
\lstinputlisting[caption={Griewank Function}, label={lst:pso_fun}]{../griewank_fun.m}
\lstinputlisting[caption={Particle Swarm}, label={lst:pso}]{../../OptimModule/optimizers/global/particleswarm.m}
\subsection*{Genetic Algorithm Code}
\lstinputlisting[caption={GA Main Code},label={lst:ga_main}]{../../OptimModule/optimizers/global/GA/GA_main.m}
\lstinputlisting[caption={Fitness function},label={lst:ga_fit}]{../../OptimModule/optimizers/global/GA/fitness.m}
\lstinputlisting[caption={Encoding function},label={lst:ga_encode}]{../../OptimModule/optimizers/global/GA/encode.m}
\lstinputlisting[caption={Decoding function},label={lst:ga_decode}]{../../OptimModule/optimizers/global/GA/decode.m}
\lstinputlisting[caption={Roulette-wheel selection function},label={lst:ga_selection_roulette}]{../../OptimModule/optimizers/global/GA/roulette.m}
\lstinputlisting[caption={Tournament selection function},label={lst:tournament_selection}]{../../OptimModule/optimizers/global/GA/tournament_selection.m}
\lstinputlisting[caption={Cross-over function},label={lst:ga_xover}]{../../OptimModule/optimizers/global/GA/crossover.m}
\lstinputlisting[caption={Elitism function},label={lst:ga_elitism}]{../../OptimModule/optimizers/global/GA/elitism.m}
\lstinputlisting[caption={Logging function},label={lst:ga_logger}]{../../OptimModule/optimizers/global/GA/log_f.m}
\begin{lstlisting}[caption={Output},label={lst:op}]

x_star_min =

    0.0007
    0.0023


fval =

   1.6217e-06


x_star_max =

    0.0000
    4.4474


fval =

    2.0049

-----------Linear Programming----------------
Optimal solution found.


x_star_linprog =

   15.0000
         0
    3.3333
         0


ans =

  113.3333
\end{lstlisting}
\end{document}
